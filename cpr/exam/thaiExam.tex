\documentclass[a4paper,16pt]{examdesign}
% Using examdesign package from CTAN

\NumberOfVersions{3}
% \NoRearrange
\BoldfaceCorrectMultipleChoiceAnswer
\class{980-022 Basic Lifesaving}
\examname{Written test}
\setrandomseed{1024}
\def\namedata{}



\usepackage{fontspec}
\XeTeXlinebreaklocale "th_TH"
\defaultfontfeatures{Scale=2}
\setmainfont{TH Sarabun New}

\usepackage[margin=1in]{geometry}
\usepackage{enumitem}

\setlist[enumerate,3]{label=\Alph*}


\begin{document}


\begin{frontmatter}
	\vspace*{2in}
	\begin{center}
		{\huge 980-022 Basic Lifesaving} \\
		\vspace{1in}
		{\huge Written Test, ภาษาไทย} \\
		\vspace{1in}
		{\huge Version \framebox{\Alph{version}}} \\
		\vspace{1in}
		\begin{itemize}
		\item ข้อสอบนี้มีทั้งหมดรวม 12 หน้า กรุณาตรวจสอบว่าคุณได้รับทุกหน้า 
		\item ห้ามเขียนลงบนข้อสอบ ฝนคำตอบบนกระดาษคำตอบเท่านั้น
		\item ระบาย Version ของข้อสอบบนกระดาษคำตอบด้วย
		\item ข้อสอบใช้เวลา 1 ชั่วโมง
		\item ข้อสอบเป็นแบบปรนัย จำนวน 50 คำถาม
		\item กรุณาเช็คการฝน ID และ version ของข้อสอบบนกระดาษคำตอบก่อนส่ง
		\end{itemize}
	\end{center}
	\vfill
\end{frontmatter}


\begin{examtop}
	{\parbox{3in}{\classdata \\
	\examtype, Version: \fbox{\textsf{\Alph{version}}}}}
\end{examtop}

% \begin{multiplechoice}[keycolumns=5]
\begin{multiplechoice}
	\begin{question}การปฐมพยาบาลเบื้องต้น หมายถึง อะไร ?
		\choice[!]{การช่วยเหลือผู้ที่ได้รับบาดเจ็บ ณ สถานที่นั้น โดยใช้อุปกรณ์เท่าที่มีในขณะนั้น}
		\choice{ช่วยเหลือผู้ประสบภัยทางน้ำ}
		\choice{ช่วยเหลือผู้ที่ประสบอุบัติเหตุในโรงพยาบาลคนแรก}
		\choice{ช่วยเหลือผู้ที่ประสบภัยทางอากาศ}
		\end{question}

		\begin{question}ข้อใดเป็นสิ่งที่นักศึกษา ต้องระมัดระวังเป็น อันดับแรก ในเข้าการช่วยเหลือผู้บาดเจ็บ
			\choice{ความปลอดภัยของคนรอบข้าง}
			\choice{ความปลอดภัยของผู้บาดเจ็บ}
			\choice[!]{ความปลอดภัยของตนเอง}
			\choice{เข้าช่วยทันที เพราะต้องแข่งกับเวลา}
		\end{question}

		\begin{question}ปัจจุบัน หมายเลขโทรศัพท์ สำหรับเรียกรถพยาบาล กรณีเจ็บป่วยฉุกเฉิน ทั่วประเทศ คือข้อใด ?
			\choice{1699}
			\choice[!]{1669}
			\choice{191}
			\choice{1724}
			\end{question}
			
			\begin{question}เมื่อพบเหตุการณ์ฉุกเฉิน นักศึกษาจะขอความช่วยเหลือ ต้องแจ้งรายละเอียดอะไรบ้าง ?  (เป็นอย่างน้อย)
			\choice{สถานที่เกิดเหตุ เพศ อายุ สัญชาติ จำนวนผู้บาดเจ็บ}
			\choice{เกิดเหตุอะไร?  สถานที่เกิดเหตุ  จำนวนผู้บาดเจ็บ หมายเลขติดต่อกลับ}
			\choice[!]{เกิดเหตุอะไร? อาการของผู้บาดเจ็บ สถานที่เกิดเหตุ  หมายเลขติดต่อกลับ}
			\choice{สถานที่เกิดเหตุ อาการของผู้บาดเจ็บ เวลาที่เกิดเหตุ หมายเลขติดต่อกลับ}
			\end{question}
			
			\begin{question}เมื่อนักศึกษา ต้นแขนข้างซ้ายไปกระแทกขอบประตู มีแผลฟกช้ำ ควรให้การปฐมพยาบาลอย่างไร ? 
			\choice{ใช้ความร้อนประคบใน 24 ชม.แรก หลังเกิดอุบัติเหตุและความเย็นประคบหลังจาก 24 ชม.แรก}
			\choice[!]{ใช้ความเย็นประคบใน 24 ชม.แรก หลังเกิดอุบัติเหตุและความร้อนประคบหลังจาก 24 ชม. ต่อมา}
			\choice{ใช้ยานวดแบบร้อนทาบริเวณที่ฟกช้ำและบวม}
			\choice{บีบนวดคลึงเบาๆที่รอยช้ำเพื่อให้หายเร็ว}       
			\end{question}

				\begin{question}
				ข้อใดเป็นคำแนะนำที่เหมาะสม สำหรับการปฐมพยาบาล ผู้บาดเจ็บที่มีกระดูกหัก
				\choice{กรณีแขนหักให้ขยับมือเอาไว้ตลอดเพื่อให้มีการไหลเวียนของเลือด}
				\choice{กรณีกระดูกหักแบบเปิด ให้รีบดามกระดูกก่อนแล้วจึงปิดแผลห้ามเลือด}
				\choice[!]{กรณีกระดูกหัก ห้ามดื่มหรือรับประทานอะไรเด็ดขาด ก่อนไปถึงรพ.}                                                    
				\choice{การบาดเจ็บที่แค่บวมไม่มีการผิดรูปของกระดูก แปลว่าไม่การหักของกระดูกอย่างแน่นอน}
				\end{question}
				
				\begin{question}เมื่อพบบาดแผลที่มีวัสดุปักคา อยู่บริเวณหน้าอก  นักศึกษาจะให้การปฐมพยาบาลอย่างไร?
				\choice{รีบดึงออกแล้วทำการห้ามเลือด}
				\choice{รีบดึงออกแล้วหาหลอดมาใส่ระบายเลือด}
				\choice{ใช้มือจับวัสดุให้อยู่นิ่งๆ พร้อมพันแผลไว้รอบลำคัว}
				\choice[!]{ใช้ผ้าพันโคนวัสดุไว้ ตรึงให้วัสดุที่ปักคาอยู่นิ่งๆ}
				\end{question}			
	
				\begin{question}ข้อใด เป็นการปฐมพยาบาลแผลไฟไหม้น้ำร้อนลวก ได้ถูกต้อง
				\choice{กรณีที่แผลเป็นถุงน้ำขนาดใหญ่ให้ทำให้แตกแล้วเปิดแผลทิ้งไว้ให้แห้งหายไปเอง}
				\choice{ใช้ยางว่านหางจระเข้ ทาลงบนบาดแผล}
				\choice[!]{ใช้น้ำสะอาด ให้น้ำไหลผ่าน/ราดที่บาดแผล นานอย่างน้อย 10 นาที  หรือจนหายจากอาการปวดแสบปวดร้อน}
				\choice{ใช้ยาสีฟัน ขี้เถ้าหรือเกลือทาแผลทันที หลังโดนน้ำร้อนลวก}
				\end{question}
			
				\begin{question}ข้อใดกล่าวถูกต้อง เกี่ยวกับการขันชะเนาะ (Tourniquet) 
					\choice{ทำในทุกแผลที่มีเลือดไหลเพราะเลือดจะได้หยุดไหลเร็วขึ้น}
					\choice[!]{ทำในกรณีแผลใหญ่ เลือดไหลจำนวนมาก  หรือห้ามเลือดด้วยวิธีอื่นไม่ได้ผล}
					\choice{สามารถทำได้ทุกตำแหน่งของร่างกาย}
					\choice{ใช้วัสดุเส้นเล็กๆ เมื่อรัดแล้วควรคลายทุก 15 นาที}
					\end{question}
					
					\begin{question}ถ้านักศึกษา ถูกกระจกบาด แผลกว้างและลึก มีเลือดออกจำนวนมาก ต้องปฐมพยาบาลตามข้อใดจึงเหมาะสมที่สุด
					\choice{ใช้ผ้าสะอาดกดห้ามเลือด ประมาณ 15-20 นาที แล้วเอาผ้าออก}
					\choice[!]{ใช้ผ้าสะอาดกดห้ามเลือด รีบส่งโรงพยาบาลทันที}
					\choice{ใช้ผ้าสะอาดกดห้ามเลือด แล้วใช้น้ำทำความสะอาดบาดแผล}
					\choice{ใช้ผ้าสะอาดกดที่ปากแผลเพื่อห้ามเลือด แล้วใช้น้ำเกลือทำความสะอาด}
					\end{question}
					
					\begin{question}นิ้วชี้ขวาของเพื่อนถูกมีดตัดขาด นักศึกษาจะดูแลอวัยวะที่ถูกตัดขาดของเพื่อน อย่างไร?
					\choice{รีบนำไปแช่ในน้ำแข็งทันที}
					\choice{ห่อผ้าสะอาดไว้ แล้วแช่ในน้ำแข็ง}
					\choice[!]{ห่อด้วยผ้าสะอาดแล้วใส่ถุงพลาสติกหรือวัสดุที่น้ำไม่เข้า มัดปากถุงให้สนิทก่อนแช่ในน้ำแข็ง}
					\choice{นำไปแช่ในช่องทำน้ำแข็ง}
					\end{question}

					\begin{question}นักศึกษาไปเที่ยวทะเลกับเพื่อน  เพื่อนถูกแมงกระพรุนบริเวณต้นขา ปวดแสบปวดร้อนมาก นักศีกษาจะให้การปฐมพยาบาลอย่างไร?
						\choice[!]{ราดด้วยน้ำส้มสายชู แล้วใช้อุปกรณ์แบนๆ ครูดคราบเมือกออก}
						\choice{ใช้น้ำเปล่าราดมากๆ จนคราบเมือกออกหมด}
						\choice{ใช้ทรายขยี้ให้คราบเมือกหลุดออก}
						\choice{ใช้ผักบุ้งทะเลพอก โดยไม่ต้องล้างคราบเมือก}
						\end{question}
						
						\begin{question}เมื่อนักศึกษา ถูกผึ้งต่อย ข้อใด?ปฐมพยาบาลได้ถูกต้อง
						\choice{ใช้ผ้ายืดพันทับ ลดอาการบวม}
						\choice[!]{ดึงเหล็กในออก แล้วประคบด้วยความเย็น}
						\choice{ประคบความร้อน เพื่อลดอาการบวม}
						\choice{ใช้อุปกรณ์ทุบ แล้วประคบความร้อน}
						\end{question}
						
						\begin{question}เมื่อเพื่อนของนักศึกษาไปเที่ยวทะเล ถูกเม่นทะเลตำเท้า จะให้การปฐมพยาบาลอย่างไร?
						\choice{รัดเหนือแผลป้องกันพิษเข้าสู่หัวใจ}
						\choice{ใช้รองเท้าทุบแรงๆให้เข็มแตกให้ละเอียด}
						\choice[!]{บีบน้ำมะนาวใส่แผล หรือแช่น้ำร้อนผสมน้ำส้มสายชู}
						\choice{แช่น้ำร้อน 45 C นาน  3  ชั่วโมง}
						\end{question}	

						\begin{question}ถ้านักศึกษา พบสถานการณ์ฉุกเฉิน และพบผู้ป่วยไม่รู้สึกตัว ไม่หายใจ ข้อใดเป็นเวลานาทีทองของการช่วยชีวิต 
							\choice{3 นาที}
							\choice[!]{4 นาที}
							\choice{5 นาที}
							\choice{6 นาที}
							\end{question}
							
							\begin{question}ขณะที่นักศึกษากำลังวิ่งออกกำลังกายในสวนสาธารณะ พบผู้ป่วยนอนหมดสติอยู่ที่พื้นสนาม  หลังจากตรวจสอบความปลอดภัยโดยรอบแล้ว ข้อใดที่ต้องปฏิบัติ เป็นอันดับแรก
							\choice{ตรวจสอบว่าผู้ป่วยยังหายใจหรือไม่  แล้วรีบทำการช่วยฟื้นคืนชีพทันที (CPR)}
							\choice[!]{ปลุกเรียกผู้ป่วย เพื่อตรวจสอบความรู้สึกตัว และรีบขอความช่วยเหลือ}
							\choice{ร้องขอความช่วยเหลือ และโทรเรียกรถพยาบาลฉุกเฉิน}
							\choice{ทำการช่วยฟื้นคืนชีพทันที (CPR)  และขอเครื่อง AED}
							\end{question}
							
							\begin{question}ข้อใด ไม่ใช่ ปัจจัยเสี่ยง ของโรคหลอดเลือดหัวใจ
							\choice{ความดันโลหิตสูง}
							\choice{สูบบุหรี่เป็นประจำ}
							\choice[!]{ดื่มไวน์ 1 แก้ว นานๆ ครั้ง}
							\choice{มีภาวะไขมันในเลือดสูง}
							\end{question}
							
							\begin{question}ข้อใด เป็นกลุ่มอาการที่มีความเสี่ยง โรคกล้ามเนื้อหัวใจขาดเลือดไปเลี้ยงเฉียบพลัน
							\choice{ปวดเมื่อยไหล่ด้านซ้ายเวลาขยับหลังเล่นเทนนิส}
							\choice{ปวดเกร็ง บีบๆท้อง เป็นๆหายๆ มาเป็นเวลา 2 อาทิตย์}
							\choice[!]{แน่นกลางหน้าอกร้าวขึ้นมาบริเวณกราม ใจสั่นและเหนื่อยง่าย เมื่อเวลาออกแรง}
							\choice{เจ็บใต้ชายโครงขณะไอแรงๆและมีไข้}
							\end{question}
							
							\begin{question}ข้อใด ไม่ใช่ การช่วยเหลือเบื้องต้นสำหรับผู้ที่มีอาการ โรคหลอดเลือดหัวใจกำเริบ
							\choice{ถ้าผู้ป่วยหมดสติ ไม่หายใจ  ให้เริ่มทำ CPR}
							\choice{จัดท่าให้ผู้ป่วยอยู่ในท่านั่งหรือนอนให้สบาย}
							\choice{ถ้าผู้ป่วยมียาขยายหลอดเลือดกินประจำ ให้จัดยาให้กิน}
							\choice[!]{ทำท่ารัดอัดบริเวณท้อง}
							\end{question}
							
							\begin{question}ข้อใดเป็นกลุ่มอาการ ที่ใช้ประเมินโรคหลอดเลือดสมอง                                                                                     
							\choice[!]{B. E. F. A. S. T.}
							\choice{B. E. S. T. R. O. N. G.}
							\choice{B. E. F. I. R. S. T.}
							\choice{B. E. S. M. A. R. T.}
							\end{question}
							
							\begin{question}ภาวะความดันโลหิตสูง เสี่ยงให้เกิดโรคหรือ อาการใดมากที่สุด
							\choice{โรคหลอดเลือดในสมองตีบ}
							\choice{โรคหลอดเลือดในสมองอุดตัน}
							\choice[!]{โรคหลอดเลือดในสมองแตก}
							\choice{โรคหลอดเลือดแดงแข็ง}
							\end{question}
							
							\begin{question}ข้อใด? ไม่ใช่  อาการแสดงของการแพ้รุนแรง
							\choice[!]{มีความดันโลหิตสูง}
							\choice{มีความดันโลหิตต่ำ}
							\choice{มีผื่น คัน ปากบวม หน้าบวม}
							\choice{มีหอบเหนื่อย หลอดลมตีบ}
							\end{question}
							
							\begin{question}ข้อใด? ไม่ใช่ สาเหตุของการเป็นลม
							\choice{อากาศร้อน หิวข้าว มีระดับน้ำตาลในเลือดต่ำ}
							\choice{โรคหลอดเลือดหัวใจ}
							\choice{ความดันโลหิตต่ำ สูญเสียเลือด}
							\choice[!]{ความดันโลหิตสูง}
							\end{question}
				
							
				\begin{question}นักศึกษาจะปฐมพยาบาลผู้ป่วยเป็นลม อย่างไร?
							\choice{จัดท่าให้ผู้ป่วยนอนศีรษะสูง}
							\choice{รีบให้ผู้ป่วยรับประทานยาลม หรือรีบให้ดื่มน้ำหวานทันที}
							\choice[!]{จัดให้ผู้ป่วยนอนราบศีรษะต่ำ ใช้ผ้าชุบน้ำเช็ดใบหน้า}
							\choice{เมื่อผู้ป่วยเริ่มรู้สึกตัวให้จัดท่านั่ง และลุกเดินได้ทันที}
							\end{question}
							
							\begin{question}ผู้ป่วยในกลุ่มใด ?  ที่แพทย์แนะนำให้รับประทานยา เพื่อป้องกันอาการชัก
							\choice{ผู้ที่ชักจากความเหน็ดเหนื่อยในการทำงาน}
							\choice{ผู้ที่ชักจากภาวะน้ำตาลในเลือดต่ำ}
							\choice[!]{ผู้ที่เคยได้รับการผ่าตัดสมอง}
							\choice{ผู้ที่เคยชักจากการได้รับยากระตุ้น}
							\end{question}
							
							\begin{question}ข้อใดเป็นการดูแลผู้ที่มีอาการชัก ได้ถูกต้องที่สุด
							\choice{จัดท่าให้นอนราบ หาช้อนใส่ปากป้องกันการกัดลิ้น ดูแลสิ่งแวดล้อมไม่ให้เกิดอันตราย}
							\choice{จัดท่าให้นั่ง หาช้อนใส่ปากป้องกันการกัดลิ้น ดูแลสิ่งแวดล้อมไม่ให้เกิดอันตราย}
							\choice{จัดท่าให้นั่ง ดูแลสิ่งแวดล้อมไม่ให้เกิดอันตราย  ป้องกันการบาดเจ็บ อยู่กับผู้ป่วยตลอดเวลา}
							\choice[!]{จัดท่าให้นอนราบ ดูแลสิ่งแวดล้อมไม่ให้เกิดอันตราย  ป้องกันการบาดเจ็บ อยู่กับผู้ป่วยตลอดเวลา}
							\end{question}
							
							\begin{question}ข้อใด ?  คืออาการบ่งชี้ ในผู้ที่มีอาการอุดกั้นทางเดินหายใจเฉียบพลัน 
							\choice{เอามือกุมที่คอ ร้องไม่มีเสียง กระสับกระส่าย}
							\choice{ร้องขอความช่วยเหลือ บอกหายใจไม่ออก ริมฝีปากเขียวคล้ำ}
							\choice{ชัก หมดสติ ไม่รู้สึกตัว}
							\choice[!]{ถูกเฉพาะข้อ ก  และ  ค}
							\end{question}

							\begin{question}นักศึกษาพบเด็กทารกอายุ 10 เดือน กลืนสิ่งแปลกปลอม ไม่มีเสียงร้อง หายใจมีเสียงวี้ดๆ กระสับกระส่าย จะให้การช่วยเหลือ อย่างไร?
								\choice{ทำการกดกระทุ้งท้อง  สลับการกดหน้าอก 5  ครั้ง}
								\choice{รีบล้วงคอ/ ใช้นิ้วก้อย เอาสิ่งแปลกปลอมออกมาให้เร็วที่สุด}
								\choice[!]{จับนอนคว่ำ ศีรษะต่ำเล็กน้อย ตบหลังบริเวณกึ่งกลางสะบัก 5 ครั้ง แล้วสลับกับการกดหน้าอก 5 ครั้ง}
								\choice{จับนอนหงายแล้ว CPR ทันที}
								\end{question}
								
								\begin{question}หลังจากปฐมพยาบาลทารกที่กลืนสิ่งแปลกปลอมด้วยการตบหลังและกดหน้าอก พบว่าทารกหมดสติในเวลาต่อมา ไม่ตอบสนอง นักศึกษาจะให้การช่วยเหลืออย่างไร?                              
								\choice[!]{เริ่มทำ CPR และมองดูในปากว่า มีสิ่งแปลกปลอมหลุดออกมา หรือไม่}
								\choice{ตบหลังบริเวณกึ่งกลางสะบัก 5 ครั้ง แล้วสลับกับการกดหน้าอก 5 ครั้ง}
								\choice{ทำการกดกระทุ้งท้อง 5 ครั้ง สลับการกดหน้าอก 5 ครั้ง}
								\choice{ทำการกดกระทุ้งท้อง 5 ครั้ง  สลับการตบหลังบริเวณกึ่งกลางสะบัก 5 ครั้ง}
								\end{question}
								
								\begin{question}ปฏิบัติการช่วยฟื้นคืนชีพ (Cardio Pulmonary Resuscitation) หมายถึงอะไร?
								\choice{การนวดผ่อนคลาย เพื่อให้ผู้ป่วยรู้สึกตัว}
								\choice[!]{การช่วยเหลือผู้ที่หยุดหายใจ หรือหัวใจหยุดเต้นให้มีการหายใจและการไหลเวียนกลับคืนสู่สภาพเดิม ป้องกันเนื้อเยื่อได้รับอันตรายจากการขาดออกซิเจนอย่างถาวร}
								\choice{เหตุการณ์ที่ไม่คาดฝันและต้องมีการช่วยเหลือผู้ป่วยหรือผู้บาดเจ็บ}
								\choice{ถูกทุกข้อ}
								\end{question}
						 
					 \begin{question}ข้อใด ? คือวัตถุประสงค์ของการช่วยฟื้นคืนชีพ
							\choice{เพิ่มออกชิเจนให้เนื้อเยื่อ}
							\choice{ป้องกันภาวะสมองขาดเลือด}
							\choice{คงไว้ซึ่งการไหลเวียนโลหิต}
							\choice[!]{ถูกทุกข้อ}
							\end{question}
							
							\begin{question}ข้อใด ? เป็นสาเหตุสำคัญ ของการหยุดหายใจ
							\choice{การถูกกระแสไฟฟ้าแรงสูงดูด}
							\choice{การได้รับยาเกินขนาดหรือการแพ้รุนแรง}
							\choice{หัวใจวายจากโรคหัวใจจากการออกกำลังกาย มากเกินปกติ}
							\choice[!]{ทางเดินหายใจอุดกั้นเฉียบพลัน ทำให้ร่างกายได้รับออกชิเจนไม่เพียงพอ}
							\end{question}
							
							\begin{question}อะไร? เป็นสาเหตุสำคัญ ของหัวใจหยุดเต้น
							\choice{การถูกกระแสไฟฟ้าช๊อต}
							\choice{ได้รับยาเกินขนาด}
							\choice{มีอาการแพ้รุนแรง}
							\choice[!]{ถูกทุกข้อ}
							\end{question}
							
							\begin{question}นักศึกษาพบเห็นคนหมดสติ เรียกไม่ตอบสนอง มีการหายใจเฮือก ควรให้ความช่วยเหลืออย่างไร?
							\choice{เฝ้าดูอาการอย่างใกล้ชิด เนื่องจากการหายใจเฮือก เป็นการหายใจที่ปกติ}
							\choice{ทำการช่วยหายใจเพียงอย่างเดียว  เนื่องจากการหายใจเฮือก ไม่ใช่การหายใจปกติ}
							\choice[!]{เริ่มทำ CPR  ทันที เนื่องจากการหายใจเฮือก ไม่ใช่การหายใจปกติ}
							\choice{เริ่มทำ CPR  แม้ว่าการหายใจเฮือก จะเป็นการหายใจที่ปกติ}
							\end{question}
							
							\begin{question}ข้อใด ? ไม่ถูกต้องในการช่วยฟื้นคืนชีพ 
							\choice{ทำเฉพาะใน ผู้ที่หมดสติ ปลุกเรียก ไม่รู้สึกตัว  ไม่หายใจ  ไม่มีชีพจร เท่านั้น}
							\choice{การวางตำแหน่งมือผิด  จะมีความเสี่ยงทำให้ซี่โครงหัก และมีการบาดเจ็บของอวัยวะภายในได้}
							\choice[!]{การกดหน้าอกให้ตรงตามจังหวะอารมณ์ ของผู้ช่วยเหลือแบบใดก็ได้ เพราะไม่สำคัญ}
							\choice{ในขณะกดหน้าอก จะไม่หยุดกด เกิน 10  วินาที}
							\end{question}
							
							\begin{question}ข้อใด ? คือ การกดหน้าอกอย่างมีประสิทธิภาพ (High Quality  CPR)
							\choice{วางมือในตำแหน่งใดก็ได้ บริเวณหน้าอก}
							\choice[!]{กดหน้าอกให้ลึก มากกว่า 2 – 2.4 นิ้ว หรือ 5 – 6 ซม. และปล่อยให้หน้าอกคืนตัวสุด}
							\choice{กดหน้าอกด้วยอัตราเร็ว  80 – 120  ครั้งต่อนาที}
							\end{question}
							
							\begin{question}เพราะเหตุใด ? การปล่อยหน้าอกให้คืนตัวกลับอย่างเต็มที่ หลังจากการกดหน้าอกแต่ละครั้ง จึงมีความสำคัญในการทำ CPR  อย่างมีประสิทธิภาพ
							\choice{สามารถลดความเหนื่อยล้าของผู้ช่วยเหลือได้}
							\choice{สามารถลดความเสี่ยงการเกิดกระดูกซี่โครงหัก}
							\choice[!]{หัวใจสามารถคลายตัวรับเลือดที่ไหลกลับเข้าสู่หัวใจได้เต็มที่ ระหว่างการกดหน้าอก}
							\choice{ช่วยให้อัตราเร็วในการกดหน้าอก เร็วขึ้น}
							\end{question}
							
							\begin{question}ข้อใด? ต่อไปนี้จะทำให้ผู้ป่วยหัวใจหยุดเต้นเฉียบพลัน มีโอกาสรอดชีวิตมากที่สุด                                          
							\choice{การช่วยหายใจอย่างรวดเร็ว}
							\choice{การทำ  CPR  อย่างมีประสิทธิภาพ}
							\choice[!]{การทำ  CPR  อย่างมีประสิทธิภาพ  ร่วมกับการใช้เครื่องกระตุกไฟฟ้าหัวใจอัตโนมัติ}
							\choice{การทำ CPR ด้วยอัตราเร็วที่สุดเท่าที่จะทำได้}
							\end{question}
							
							\begin{question}
							นักศึกษา จะทำ CPR  บุคคลใด ต่อไปนี้
							\choice[!]{ชายไทยอายุ  65  ปี  นอนหมดสติ ปลุกเรียก ไม่รู้สึกตัว  ไม่หายใจ}
							\choice{หญิงไทยอายุ  42  ปี ประสบอุบัติเหตุ  ปลุกเรียกรู้สึกตัว หายใจ 16  ครั้ง/ นาที  มีบาดแผลเลือดออกจำนวนมาก}
							\choice{ชายไทยอายุ  50 ปี  หายใจ 32  ครั้ง/ นาที  กระสับกระส่าย พูดคุยสับสน}
							\choice{เด็กชาย อายุ  10 ขวบ  ถูกงูกัด เรียกลืมตา หายใจ 12  ครั้ง/ นาที}
							\end{question}
							
							\begin{question}ข้อใด ?  หมายถึงเครื่อง AED                                                                                                             
							\choice{เครื่องตรวจวัดออกซิเจน}
							\choice{เครื่องกระตุ้นหัวใจ}
							\choice[!]{เครื่องกระตุกหัวใจด้วยไฟฟ้าอัตโนมัติ}
							\choice{เครื่องตรวจวัดสัญญาณชีพ}
							\end{question}
							
							\begin{question}
							นักศึกษาจะใช้เครื่อง AED กับผู้ป่วยรายใด?
							\choice{คนที่รู้สึกตัว แต่หายใจช้า}
							\choice[!]{คนที่บ่นเจ็บหน้าอก แล้วหมดสติ ปลุกเรียกไม่รู้สึกตัว ไม่หายใจ}
							\choice{คนที่ชักเกร็งกระตุก ไม่รู้สึกตัว หายใจเร็ว}
							\choice{คนเมาที่หมดสติ หายใจเสียงดัง}
							\end{question}
							
							\begin{question}นักศึกษาจะใช้เครื่อง AED เมื่อใด?
							\choice{เมื่อผู้ป่วยหยุดชัก และเริ่มกลับมาหายใจ}
							\choice{เมื่อ CPR ครบ 2 นาที หรือ 5 รอบ}
							\choice[!]{ใช้ทันทีที่เครื่อง AED มาถึง}
							\choice{เมื่อทีมแพทย์ฉุกเฉินมาถึง}
							\end{question}
							
							
							\begin{question}ขณะนักศึกษา กำลังทำ CPR ผู้ที่ประเมินว่า ไม่รู้สึกตัวไม่หายใจ เพื่อนนำเครื่อง AED  มาถึงตัวผู้ป่วย ขั้นตอนแรกที่ต้องทำ คืออะไร ?
							\choice{กดปุ่มช็อกไฟฟ้า}
							\choice{ติดแผ่นช็อกไฟฟ้า (AED pads) บนหน้าอกของผู้ป่วย}
							\choice{ประกาศ “ห้ามสัมผัสผู้ป่วย”}
							\choice[!]{กดปุ่มเปิดใช้งานเครื่อง AED}
							\end{question}

							\begin{question}ข้อใด? ถูกต้องเกี่ยวกับการใช้งานเครื่อง AED 
								\choice{สามารถใช้เครื่อง AED ได้กับผู้ป่วยขณะที่ยังจมอยู่ในน้ำ}
								\choice{ก่อนติดแผ่นช็อกไฟฟ้า ห้ามนำแผ่นยาที่ติดอยู่บริเวณหน้าอกผู้ป่วยออก}
								\choice{ห้ามใช้เครื่อง AED กับผู้ป่วยที่มีเครื่องกระตุ้นไฟฟ้าหัวใจชนิดฝัง อยู่บริเวณหน้าอก}
								\choice[!]{ในผู้ป่วยที่มีขนหน้าอกมาก แผ่นช็อกไฟฟ้าอาจไม่ติดแนบกับหน้าอกผู้ป่วย และอาจทำให้ไม่สามารถทำการช็อกไฟฟ้าหัวใจได้}
								\end{question}
								
								\begin{question}นักศึกษาต้องทำอย่างไร? หลังจากทำการ ช็อกไฟฟ้าด้วยเครื่อง AED แล้ว
								\choice{คลำชีพจรด้วยความรวดเร็ว ภายในเวลาไม่เกิน 10 วินาที}
								\choice[!]{เริ่มทำ CPR โดยเริ่มจากการกดหน้าอก ทันที}
								\choice{ทำการช่วยหายใจ 2 ครั้ง}
								\choice{รอจนกว่าเครื่อง AED จะทำการวิเคราะห์จังหวะการเต้นของหัวใจซ้า}
								\end{question}
								
								\begin{question}ความสำคัญของการใช้ AED คืออะไร ? 
								\choice{ป้องกันการเกิดภาวะหัวใจหยุดเต้นซ้ำได้}
								\choice{ไม่ได้มีความสำคัญสำหรับภาวะหัวใจหยุดเต้น แต่อย่างใด}
								\choice[!]{สามารถทำให้คลื่นไฟฟ้าหัวใจ ที่เต้นผิดจังหวะกลับมาเป็นจังหวะปกติได้}
								\choice{สามารถรักษาให้คลื่นไฟฟ้าหัวใจกลับมาเป็นปกติได้ในผู้ป่วยภาวะหัวใจหยุดเต้นทุกราย}
								\end{question}
								
								\begin{question}หลังจากนักศึกษา ติดแผ่นแปะเครื่องช็อกไฟฟ้าหัวใจ บนหน้าอกของผู้ป่วยที่ยังคงไม่รู้สึกตัว ไม่หายใจ ขั้นตอนต่อไปในการใช้เครื่อง AED คือ 
								\choice{ประกาศ “ทุกคนถอยห่าง เตรียมช็อก”}
								\choice{ตรวจดูว่าระดับความรู้สึกตัว และการหายใจของผู้ป่วย}
								\choice[!]{ทำตามที่เครื่อง AED สั่ง}
								\choice{กดปุ่มช็อกไฟฟ้าทันที}
								\end{question}
								
								\begin{question}ในการใช้เครื่อง AED ก่อนนักศึกษากดปุ่มช็อก ขั้นตอนใด ?สำคัญมาก
								\choice{เตรียมกดปุ่มช็อกไฟฟ้าทันที ภายในระยะเวลาอันรวดเร็ว}
								\choice[!]{สายตามองที่ผู้ป่วย นิ้วเตรียมกดที่ปุ่มช็อก ประกาศ “ทุกคนถอยห่าง เตรียมช็อก”}
								\choice{ตรวจดูระดับความรู้สึกตัว และการหายใจของผู้ป่วย}
								\choice{ทำตามที่เครื่อง AED สั่ง}
								\end{question}
								
								\begin{question}
								นักศึกษาพบ ชายต่างชาติอายุ 67 ปี นอนหมดสติ ไม่หายใจ  นักศึกษาและผู้ช่วยเหลืออีกคนกำลังทำ CPR อย่างมีประสิทธิภาพ เมื่อใดที่นักศึกษาและผู้ช่วยเหลืออีกคน ควรสลับหน้าที่ในระหว่างทำ CPR                             
								\choice[!]{สลับหน้าที่กันทุก 2 นาที หรือเมื่อเหนื่อย}
								\choice{สลับหน้าที่กันทุก 5 นาที}
								\choice{สลับหน้าที่กัน เมื่อมีการติดแผ่นช็อกไฟฟ้าหัวใจของเครื่อง AED}
								\choice{ไม่ต้องสลับหน้าที่กันเลย ต่างคนต่างทำหน้าที่ของตนเอง}
								\end{question}
								
								% \begin{question}ข้อใด? เป็นข้อควรระวังในการใช้ เครื่อง AED
								% \choice{เตรียมกดปุ่มช็อกไฟฟ้าทันที ภายในระยะเวลาอันรวดเร็ว}
								% \choice[!]{สายตามองที่ผู้ป่วย นิ้วเตรียมกดที่ปุ่มช็อก ประกาศ “ทุกคนถอยห่าง เตรียมช็อก”}
								% \choice{ตรวจดูระดับความรู้สึกตัว และการหายใจของผู้ป่วย}
								% \choice{ทำตามที่เครื่อง AED สั่ง}
								% \end{question}
								
								\begin{question}เมื่อนักศึกษามีบทบาทสำคัญในการช่วยชีวิตผู้อื่น องค์ประกอบใด?  ถูกต้องที่สุดในการรอดชีวิตของผู้ป่วย      
								\choice{การขอความช่วยเหลือ  การรอเวลาทีมช่วยเหลือมาช่วย เพราะมีอุปกรณ์พร้อม}
								\choice{การมีเครื่องมือและอุปกรณ์ช่วยเหลือที่พร้อม  นักศึกษามีทักษะการช่วยเหลืออย่างถูกต้อง}
								\choice[!]{ห่วงโซ่การรอดชีวิต มีความรู้ในการประเมินอาการอย่างรวดเร็ว มีทักษะการ CPR อย่างมีประสิทธิภาพและรวดเร็ว มีการนำเครื่อง AED มาใช้อย่างรวดเร็ว}
								\choice{การ CPR อย่างมีประสิทธิภาพ และทำตามที่เครื่อง AED สั่ง}
								\end{question}
\end{multiplechoice}

%\begin{endmatter}
%	\vfill
%	\vspace*{2in}
%	\begin{center}
%		{\huge This page is intentionally left blank.} \\
%	\vfill
%	\end{center}
%\end{endmatter}

\end{document}
