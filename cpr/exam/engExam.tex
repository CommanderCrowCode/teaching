\documentclass[a4paper,16pt]{examdesign}
% Using examdesign package from CTAN

\BoldfaceCorrectMultipleChoiceAnswer
\NoRearrange
\NumberOfVersions{1}
\class{980-022 Basic Lifesaving}
\examname{Written test}
\setrandomseed{1024}
\def\namedata{}



\usepackage{fontspec}
\defaultfontfeatures{Scale=2}
\setmainfont{TH Sarabun New}

\usepackage[margin=1in]{geometry}
\usepackage{enumitem}

\setlist[enumerate,3]{label=\Alph*}


\begin{document}


\begin{frontmatter}
	\vspace*{2in}
	\begin{center}
		{\huge 980-022 Basic Lifesaving} \\
		\vspace{1in}
		{\huge Written Test, English} \\
		\vspace{1in}
		{\huge Version \framebox{\Alph{version}}} \\
		\vspace{1in}
		\begin{itemize}
		\item This test has 12 pages including the cover. Please check that you have all the pages. 
		\item DO NOT write on the exam paper. All answers should be given in the provided answer sheet.
		\item Remember to color the version of this exam on the answer sheet.
		\item Cheating on the exam results in an automatic fail.
		\item The exam is one hour long. 
		\item The exam has 50 multiple-choice questions.
		\item Check that the bubbles for the student ID and the version of the test are colored correctly before submission.
		\end{itemize}
	\end{center}
	\vfill
\end{frontmatter}


\begin{examtop}
	{\parbox{3in}{\classdata \\
	\examtype, Version: \fbox{\textsf{\Alph{version}}}}}
\end{examtop}

% \begin{multiplechoice}[keycolumns=5]
\begin{multiplechoice}
	\begin{question}What does "basic first aid" mean?
		\choice[!]{Assisting injured individuals at the scene using the available equipment at that time}
		\choice{Assisting water accident victims}
		\choice{Assisting accident victims in the hospital first}
		\choice{Assisting air accident victims}
		\end{question}
		
		\begin{question}What is the first thing that a student must be cautious of when providing assistance to an injured person?
		\choice{Safety of the people around the area}
		\choice{Safety of the injured person}
		\choice[!]{Their own safety}
		\choice{Provide assistance immediately because time is of the essence}
		\end{question}

		\begin{question}Currently, what is the emergency ambulance telephone number throughout the country for calling in case of emergency illness?
			\choice{1699}
			\choice[!]{1669}
			\choice{191}
			\choice{1724}
			\end{question}
			
			\begin{question}When encountering an emergency situation, what details must a student provide when requesting assistance? (at least)
			\choice{Location of the incident, gender, age, nationality, number of injured persons}
			\choice{What happened? Location of the incident, number of injured persons, contact number for callback}
			\choice[!]{What happened? Symptoms of the injured person, location of the incident, contact number for callback}
			\choice{Location of the incident, symptoms of the injured person, time of the incident, contact number for callback}
			\end{question}
			
			\begin{question}If a student accidentally bumps their left arm against the door frame and has a bruise, how should they provide first aid?
			\choice{Apply heat for the first 24 hours after the accident, and apply cold after 24 hours.}
			\choice[!]{Apply cold for the first 24 hours after the accident, and apply heat after 24 hours.}
			\choice{Apply hot massage to the bruised and swollen area.}
			\choice{Gently massage and squeeze the bruised area to help it heal faster.}
			\end{question}


\begin{question}
			Which of the following is a suitable advice for first aid for a person with a broken bone?
\choice{In case of a broken arm, keep moving the hand to promote blood flow.}
\choice{In case of an open fracture, immediately compress the bone and then close the wound to prevent bleeding.}
\choice[!]{In case of a broken bone, do not eat or drink anything until you reach the hospital.}
\choice{If the injury only causes swelling without definite bone deformity, it is not a definite bone fracture.}
\end{question}

			\begin{question}When encountering a wound with a foreign object stuck in the chest area, how should a student provide first aid?
				\choice{Quickly remove the object and apply direct pressure to the wound}
				\choice{Quickly remove the object and insert a tube to drain blood}
				\choice{Use your hand to hold the object still and bandage the wound around the torso}
				\choice[!]{Use a cloth to wrap around the base of the object and secure it in place, keeping the object still}
				\end{question}

				\begin{question}Which of the following is the correct first aid for a burn caused by hot water?
					\choice{If the blister is large, burst it and let the wound dry out on its own}
					\choice{Apply aloe vera gel onto the wound}
					\choice[!]{Rinse the wound with clean water, letting the water run over it for at least 10 minutes or until the pain subsides}
					\choice{Apply toothpaste, lime, or salt onto the wound immediately after being burned by hot water}
					\end{question}

					\begin{question}Which of the following is correct regarding the use of a tourniquet?
						\choice{Use it on every wound with bleeding to stop bleeding faster}
						\choice[!]{Use it in cases of large wounds with excessive bleeding or when other methods to stop the bleeding are not effective}
						\choice{It can be used in any part of the body}
						\choice{Use a thin material to wrap the affected area and release the pressure every 15 minutes}
						\end{question}


						\begin{question}If a student is cut by broken glass, with a wide and deep wound and a large amount of bleeding, what is the most appropriate first aid action to take?
							\choice{Use a clean cloth to apply pressure to the wound for about 15-20 minutes, then remove the cloth.}
							\choice[!]{Use a clean cloth to apply pressure to the wound and send the student to the hospital immediately.}
							\choice{Use a clean cloth to apply pressure to the wound and clean the wound with water.}
							\choice{Use a clean cloth to apply pressure to the wound around the mouth and clean the wound with saltwater.}
							\end{question}


							\begin{question}If a friend's right index finger is cut off with a knife, how should the student take care of the severed body part?
								\choice{Immediately put it in ice water.}
								\choice{Wrap it with a clean cloth and then put it in ice water.}
								\choice[!]{Wrap it with a clean cloth, put it in a plastic bag or material that does not allow water to enter, tie the bag tightly, then put it in ice water.}
								\choice{Put it in a container of ice.}
								\end{question}
				
								\begin{question}While on a beach trip with friends, a friend is stung by a jellyfish on the leg and experiences intense pain and burning. What should the student do for first aid?
									\choice[!]{Rinse the affected area with vinegar and use a flat object to scrape off any tentacle remnants.}
									\choice{Rinse the affected area with plain water until all tentacle remnants are removed.}
									\choice{Rub sand on the affected area to remove the tentacle remnants.}
									\choice{Use seaweed to cover the affected area without washing away the tentacle remnants.}
									\end{question}
									
									\begin{question}When a student is stung by a bee, which first aid action is appropriate?
									\choice{Use an elastic bandage to wrap and reduce swelling.}
									\choice[!]{Remove the stinger, then apply a cold compress.}
									\choice{Apply a warm compress to reduce swelling.}
									\choice{Use a hammer-like device and apply heat to the affected area.}
									\end{question}
									
									\begin{question}When a student's friend is stung by a sea urchin while at the beach, what is the proper first aid treatment?
										\choice{Tie a tourniquet above the wound to prevent venom from reaching the heart.}
										\choice{Use a strong shoe to crush the spines until they are finely broken.}
										\choice[!]{Squeeze lime juice onto the wound or soak in warm water mixed with lime juice.}
										\choice{Soak in hot water at 45 degrees Celsius for 3 hours.}
										\end{question}

									\begin{question}
										If a student encounters an emergency situation and finds a patient unconscious and not breathing, which option is the golden time to perform life-saving measures?
										\choice{3 minutes}
										\choice[!]{4 minutes}
										\choice{5 minutes}
										\choice{6 minutes}
										\end{question}
										
										\begin{question}
										While a student is jogging in the park, they find an unconscious patient lying on the ground. After checking the area's safety, which action should be taken first?
										\choice{Check if the patient is breathing, then perform CPR immediately}
										\choice[!]{Attempt to wake the patient to assess consciousness and call for help}
										\choice{Yell for help and call for an ambulance}
										\choice{Perform CPR immediately and request an AED machine}
										\end{question}
										
										\begin{question}
										Which option is not a risk factor for coronary artery disease?
										\choice{High blood pressure}
										\choice{Regular smoking}
										\choice[!]{Drinking one glass of wine occasionally}
										\choice{High blood cholesterol levels}
										\end{question}
										
										\begin{question}
										Which group of symptoms carries a risk of acute coronary syndrome (ACS), which is a sudden blockage of blood flow to the heart muscle?
										\choice{Left shoulder pain when moving after playing tennis}
										\choice{Intermittent abdominal pain for 2 weeks}
										\choice[!]{Tightness in the center of the chest, radiating to the arms, sweating, and fatigue when exerting}
										\choice{Pain under the ribs while coughing and having a fever}
										\end{question}				

										\begin{question}Which of the following is not a basic aid for someone experiencing severe coronary artery disease?
											\choice{If the patient loses consciousness and stops breathing, start CPR.}
											\choice{Position the patient comfortably, either sitting or lying down.}
											\choice{If the patient is taking blood vessel dilators regularly, provide the medication.}
											\choice[!]{Perform abdominal compressions.}
											\end{question}


											\begin{question}Which of the following is a group of symptoms used to assess cerebrovascular disease?
												\choice[!]{B. E. F. A. S. T.}
												\choice{B. E. S. T. R. O. N. G.}
												\choice{B. E. F. I. R. S. T.}
												\choice{B. E. S. M. A. R. T.}
												\end{question}
												
												\begin{question}What is the most common disease or symptom associated with high blood pressure?
												\choice{Ischemic stroke}
												\choice{Hemorrhagic stroke}
												\choice[!]{Brain aneurysm}
												\choice{Atherosclerosis}
												\end{question}
												
												\begin{question}Which of the following is not a symptom of a severe allergic reaction?
												\choice[!]{High blood pressure}
												\choice{Low blood pressure}
												\choice{Rash, itching, swollen mouth or face}
												\choice{Wheezing, shortness of breath}
												\end{question}
												\begin{question}Which of the following is not a cause of fainting?
													\choice{Hot weather, hunger, low blood sugar}
													\choice{Coronary artery disease}
													\choice{Low blood pressure, blood loss}
													\choice[!]{High blood pressure}
													\end{question}
													
													\begin{question}How should a student take care of a person who is unconscious due to hyperventilation?
													\choice{Position the patient with the head elevated}
													\choice{Quickly give the patient a gas relief medicine or have them drink water}
													\choice[!]{Position the patient lying flat with the head low, use a damp cloth to wipe the face}
													\choice{As the patient begins to recover, have them sit up and walk immediately}
													\end{question}
													
													\begin{question}In which group of patients does a doctor recommend taking medication to prevent seizures?
													\choice{Patients who have seizures from overwork}
													\choice{Patients who have seizures from low blood sugar}
													\choice[!]{Patients who have had brain surgery}
													\choice{Patients who have had seizures from stimulant drugs}
													\end{question}
													
													\begin{question}Which is the most appropriate way to take care of a person with seizures?
													\choice{Position the person lying flat, use a spoon to protect their tongue, and ensure a safe environment}
													\choice{Position the person sitting, use a spoon to protect their tongue, and ensure a safe environment}
													\choice{Position the person sitting, ensure a safe environment, prevent injuries, and stay with the person at all times}
													\choice[!]{Position the person lying flat, ensure a safe environment, prevent injuries, and stay with the person at all times}
													\end{question}						

													\begin{question}Which one is a symptom in people with sudden airway obstruction?
														\choice{Hold the neck, unable to speak, fidgeting}
														\choice{Cry for help, unable to breathe, lips turn blue}
														\choice{Convulsion, unconsciousness, no response}
														\choice[!]{Options a and c are correct}
														\end{question}

														\begin{question}A student found a 10-month-old infant swallowing a foreign object, unable to cry and breathing with wheezing and restlessness. How to provide assistance?
															\choice{Perform 5 abdominal thrusts and 5 chest compressions alternately}
															\choice{Quickly remove the foreign object by turning the infant's neck/using fingers}
															\choice[!]{Place the infant face down with the head slightly lower, tap the back of the middle of the back 5 times, then alternate with 5 chest compressions}
															\choice{Perform CPR immediately with the infant lying on their back}
															\end{question}
															
															\begin{question} After performing first-aid on a choking infant by back tapping and compression, the student found that the baby was unconscious and unresponsive afterward. How to provide assistance?
															\choice[!]{Start performing CPR and check the mouth for foreign objects}
															\choice{Tap the middle of the back 5 times and alternate with 5 chest compressions}
															\choice{Perform 5 abdominal thrusts and 5 chest compressions alternately}
															\choice{Perform 5 abdominal thrusts and alternate with 5 back slaps in the middle of the back}
															\end{question}
															
															\begin{question}What does Cardio Pulmonary Resuscitation (CPR) mean?
															\choice{A massage to relax the patient}
															\choice[!]{Assisting people who have stopped breathing or whose heart has stopped beating to breathe and circulate blood back to their normal state to prevent tissue damage from prolonged oxygen deprivation}
															\choice{An unexpected event requiring emergency assistance}
															\choice{An emergency technique used to stop bleeding in patients with severe injuries}
															\end{question}

															\begin{question}Which option is the objective of life-saving assistance?
																\choice{Increasing oxygen supply to tissues}
																\choice{Preventing stroke}
																\choice{Maintaining blood flow}
																\choice[!]{All of the above}
																\end{question}

																\begin{question}What is a major cause of respiratory arrest?
																	\choice{Being exposed to high-voltage electrical currents}
																	\choice{Overdosing on medication or having a severe allergic reaction}
																	\choice{Suffering from congestive heart failure from excessive exercise}
																	\choice[!]{Sudden blockage of the airway leading to inadequate oxygen supply to the body}
																	\end{question}
																	
																	\begin{question}What is a major cause of cardiac arrest?
																	\choice{Being exposed to electrical shock}
																	\choice{Overdosing on medication}
																	\choice{Experiencing severe allergic reactions}
																	\choice[!]{All of the above}
																	\end{question}
																	
																	\begin{question}A student comes across an unresponsive person who is gasping for breath. What should they do?
																	\choice{Observe closely since gasping is normal breathing.}
																	\choice{Administer rescue breathing alone since gasping is not normal breathing.}
																	\choice[!]{Begin CPR immediately since gasping is not normal breathing.}
																	\choice{Begin CPR even if the gasping is normal breathing.}
																	\end{question}

																	\begin{question}Which option is incorrect in assisting with CPR?
																		\choice{Only perform in individuals who have lost consciousness, are unresponsive, not breathing, and have no pulse.}
																		\choice{Placing hands incorrectly may cause rib fractures and internal organ injuries.}
																		\choice[!]{Rhythm of the chest compressions does not matter as long as the rescuer applies pressure.}
																		\choice{The rescuer should not stop chest compressions for more than 10 seconds.}
																		\end{question}

																		\begin{question}Which option describes high quality CPR?
																			\choice{Place hands anywhere on the chest.}
																			\choice[!]{Perform chest compressions that are at least 2-2.4 inches or 5-6 cm deep and allow the chest to recoil completely.}
																			\choice{Perform chest compressions at a rate of 80-120 compressions per minute.}
																			\end{question}
																			
																			\begin{question}Why is it important to allow the chest to recoil completely after each compression during high quality CPR?
																			\choice{It reduces fatigue for the rescuer.}
																			\choice{It reduces the risk of rib fractures.}
																			\choice[!]{It allows the heart to fully expand and fill with blood during chest compressions.}
																			\choice{It helps to increase the rate of chest compressions.}
																			\end{question}
																			
																			\begin{question}Which option is most likely to increase a person's chances of survival when their heart suddenly stops beating?
																			\choice{Quickly performing rescue breathing.}
																			\choice{Performing high quality CPR.}
																			\choice[!]{Performing high quality CPR with an automated external defibrillator (AED).}
																			\choice{Performing CPR at the fastest rate possible.}
																			\end{question}
													
																			\begin{question}
																				Who should the student perform CPR on in the following scenarios?
																				
																				\choice[!]{A 65-year-old unconscious Thai man who is unresponsive and not breathing.}
																				\choice{A 42-year-old Thai woman who has had an accident, is responsive, breathing 16 times per minute, and has a large amount of bleeding.}
																				\choice{A 50-year-old Thai man who is breathing rapidly, talking confusedly and erratically.}
																				\choice{A 10-year-old Thai boy who has been bitten by a snake, has lost consciousness, and is breathing 12 times per minute.}
																				\end{question}
																				
																				\begin{question}
																				Which of the following refers to an AED machine?
																				
																				\choice{Oxygen monitor}
																				\choice{Heart stimulator}
																				\choice[!]{Automated external defibrillator}
																				\choice{Vital sign monitor}
																				\end{question}
																				
																				\begin{question}
																				When should the student use an AED machine on a patient?
																				
																				\choice{When the patient is conscious but breathing slowly.}
																				\choice[!]{When the patient complains of chest pain, loses consciousness, is unresponsive, and is not breathing.}
																				\choice{When the patient is convulsing and unresponsive, but breathing rapidly.}
																				\choice{When a drunk person loses consciousness and is breathing loudly.}
																				\end{question}
																				
	\begin{question}
	When should the student use an AED machine?
	
	\choice{When the patient stops convulsing and starts breathing.}
	\choice{After performing CPR for 2 minutes or 5 cycles.}
	\choice[!]{As soon as the AED machine arrives.}
	\choice{When the emergency medical team arrives.}
	\end{question}
											
	\begin{question}
		While a student is performing CPR on a person who is unresponsive and not breathing, a friend brings an AED to the patient. What is the first step that needs to be taken?
		\choice{Press the shock button.}
		\choice{Attach the AED pads to the patient's chest.}
		\choice{Announce "Do not touch the patient."}
		\choice[!]{Press the power button to turn on the AED.}
		\end{question}
		
		\begin{question}
		Which statement is correct regarding the use of AED?
		\choice{AED can be used on a patient who is still submerged in water.}
		\choice{Before attaching the AED pads, remove any medication patches attached to the patient's chest.}
		\choice{AED cannot be used on a patient with an implanted defibrillator in their chest.}
		\choice[!]{In patients with a lot of chest hair, the AED pads may not adhere to the patient's chest and may prevent successful defibrillation.}
		\end{question}
		
		\begin{question}
		What should a student do after using an AED to deliver an electric shock?
		\choice{Check for a pulse quickly within 10 seconds.}
		\choice[!]{Start CPR by giving chest compressions immediately.}
		\choice{Provide 2 rescue breaths.}
		\choice{Wait until the AED analyzes the heart rhythm.}
		\end{question}
		
		\begin{question}
		What is the importance of using an AED?
		\choice{Preventing repeated cardiac arrest from occurring.}
		\choice{It is not important for cardiac arrest, but for other medical emergencies.}
		\choice[!]{Restoring the heart's electrical wave to a normal rhythm.}
		\choice{Being able to treat every cardiac arrest patient successfully with AED.}
		\end{question}

		\begin{question}
			After a student attaches the defibrillator pads on the chest of an unconscious and non-breathing patient, what is the next step in using an AED?
			\choice{Announce "Everyone stand clear, prepare to shock"}
			\choice{Check the patient's level of consciousness and breathing}
			\choice[!]{Follow the instructions given by the AED}
			\choice{Press the shock button immediately}
			\end{question}
			
			\begin{question}
			Before a student presses the shock button when using an AED, what is the most important step?
			\choice{Be prepared to press the shock button within a short period of time}
			\choice[!]{Look at the patient, place your fingers ready to press the shock button, and announce "Everyone stand clear, prepare to shock"}
			\choice{Check the patient's level of consciousness and breathing}
			\choice{Follow the instructions given by the AED}
			\end{question}
			
			\begin{question}
			A student finds a 67-year-old unconscious, non-breathing foreign man. The student and another assistant are performing effective CPR. When should the student and assistant switch roles during CPR?
			\choice[!]{Switch roles every 2 minutes or when tired}
			\choice{Switch roles every 5 minutes}
			\choice{Switch roles when the AED defibrillation pads are attached}
			\choice{Do not switch roles at all, each person should perform their own tasks}
			\end{question}
			
			% \begin{question}
			% Which of the following is a precaution to take when using an AED?
			% \choice{Be prepared to press the shock button within a short period of time}
			% \choice[!]{Look at the patient, place your fingers ready to press the shock button, and announce "Everyone stand clear, prepare to shock"}
			% \choice{Check the patient's level of consciousness and breathing}
			% \choice{Follow the instructions given by the AED}
			% \end{question}

			\begin{question}
				What is the most important component for students to have a significant role in saving someone's life and for the patient to survive?
				\choice{Asking for help and waiting for the rescue team because they have the necessary equipment.}
				\choice{Having ready-to-use tools and equipment, and having proper rescue skills.}
				\choice[!]{Knowing the chain of survival, having knowledge to assess the patient's condition rapidly, having effective and efficient CPR skills, and quickly using an AED machine.}
				\choice{Performing effective CPR and following the AED's instructions.}
				\end{question}

\end{multiplechoice}

%\begin{endmatter}
%	\vfill
%	\vspace*{2in}
%	\begin{center}
%		{\huge This page is intentionally left blank.} \\
%	\vfill
%	\end{center}
%\end{endmatter}

\end{document}
