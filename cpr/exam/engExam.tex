\documentclass[a4paper,16pt]{examdesign}
% Using examdesign package from CTAN

\BoldfaceCorrectMultipleChoiceAnswer
\ShortKey
\NumberOfVersions{3}
\class{980-022 Basic Lifesaving}
\examname{Written test}
\setrandomseed{1024}
\def\namedata{}



\usepackage{fontspec}
\defaultfontfeatures{Scale=2}
\setmainfont{TH Sarabun New}

\usepackage[margin=1in]{geometry}
\usepackage{enumitem}

\setlist[enumerate,3]{label=\Alph*}


\begin{document}


\begin{frontmatter}
	\vspace*{2in}
	\begin{center}
		{\huge 980-022 Basic Lifesaving} \\
		\vspace{1in}
		{\huge Written Test, English} \\
		\vspace{1in}
		{\huge Version \framebox{\Alph{version}}} \\
		\vspace{1in}
		\begin{itemize}
		\item This test has 11 pages including the cover. Please check that you have all the pages. 
		\item DO NOT write on the exam paper. All answers should be given in the provided answer sheet.
		\item Remember to color the version of this exam on the answer sheet.
		\item Cheating on the exam results in an automatic fail.
		\item The exam is one hour long. 
		\item The exam has 50 multiple-choice questions.
		\item Check that the bubbles for the student ID and the version of the test are colored correctly before submission.
		\end{itemize}
	\end{center}
	\vfill
\end{frontmatter}


\begin{examtop}
	{\parbox{3in}{\classdata \\
	\examtype, Version: \fbox{\textsf{\Alph{version}}}}}
\end{examtop}

\begin{multiplechoice}[keycolumns=5]
\begin{question}
	Which is not the goal of providing first-aid?
	\choice[!]{To fully cure the patient from the illness.}
	\choice{To reduce the chance of the patient developing long-term disabilities and chronic illness after.}
	\choice{To save the life of the the patient.}
	\choice{To reduce pain experienced by the patient and speed up the recovery.}
\end{question}

\begin{question}
	What is the phone number to call in case of medical emergency?
	\choice{1699}
	\choice[!]{1669}
	\choice{191}
	\choice{1724}
\end{question}

\begin{question}
	What information should you provide to the operator when asking for medical emergency help? (Minimum information)
	\choice{Location, sex, age, nationality, and the number of patients}
	\choice{What happened, location, the number of patients, your contact information}
	\choice[!]{What happened, symptoms of the patients, location, your contact information}
	\choice{Location, symptoms of the patients, time, your contact information}
\end{question}

\begin{question}
	For a patient in a car accident, what should be done before the patient is transported?
	\choice[!]{Always assume that the patient has a neck injury. Try to keep the patient's neck alined and not turned.}
	\choice{If the patient is outdoor, immediately move the patient inside. No first-aid is required until the patient has been moved.}
	\choice{If the patient is unconscious, put the patient in the recovery position.}
	\choice{Quickly transport to the hospital. No first-aid is required.}
\end{question}

\begin{question}
	Which is the correct recommendation on a patient with a broken bone?
	\choice{For a broken arm, continuously moving the arm to maintain blood circulation.}
	\choice{For an open fracture, splint the bone before bleeding control.}
	\choice[!]{For a broken bone, don't eat or drink anything before getting to the hospital.}
	\choice{No visibly misaligned bone means that the bone is not broken.}
\end{question}

\begin{question}
	Which is the correct method for a bruise?
	\choice[!]{Use an ice pack for the first 24 hours and then a warm pack afterward for another 24 hours.}
	\choice{Use a warm pack for the first 24 hours and then an ice pack afterward for another 24 hours.}
	\choice{Apply analgesic cream to the area.}
	\choice{Gently massage the affected area.}
\end{question}

\begin{question}
	Which is correct in splinting a broken bone?
	\choice{Realign the bone before splinting to reduce the chance of swelling.}
	\choice{No splint is needed for an open fracture. Splinting could introduce germs into the wound.}
	\choice[!]{The splint should cover from one joint below the affected area to one joint above to stabilize the area.}
	\choice{Realign the bone and immediately transfer to a hospital.}
\end{question}

\begin{question}
	What is the correct first-aid method for a dislocated shoulder?
	\choice{Realign the shoulder}
	\choice{Have the patient arms outstretched at all time}
	\choice{Let the patient rest. The shoulder will realign on its own.}
	\choice[!]{Don't realign the shoulder. Immobilize the affected arm with an arm sling.}
\end{question}

\begin{question}
	How should a stabbing wound with an object stuck inside the body be treated?
	\choice{Do not dress the wound. Have the patient hold on to the object tightly. Transport to hospital.}
	\choice{Remove the object. Control the bleeding with thick gauze.}
	\choice[!]{Immobilize the object with clean cloth. Transfer to a hospital.}
	\choice{Remove the object and put a straw in the wound. The slow drain of blood through the straw prevents pressure build-up inside the body.}
\end{question}
	
\begin{question}
	Which is correct for burn injuries?
	\choice{If a large blister develops, pop the blister. Keep the wound open until it dries.}
	\choice[!]{Wash the burnt area with plenty of clean water, at least 10 minutes or until the burning sensation subsides.}
	\choice{Rub with toothpaste, ash, or salt immediately.}
	\choice{Rub with a lot of aloe vera gel}
\end{question}

\begin{question}
	For a big cut wound caused by broken glass with a lot of bleeding, what is the correct first-aid method?
	\choice{Wash with a lot of water before dressing the wound.}
	\choice{Open up the wound and try to remove any debris before dresssing the wound.}
	\choice{Stuff the wound with clean cloth to stop the bleeding.}
	\choice[!]{Cover the wound with clean cloth. Apply pressure to stop bleeding. Cleaning is not a priority.}
\end{question}

\begin{question}
	A patient falls into a sewer drain and cut his foot. The patient is bleeding and the wound is covered with mud. If you have both water and soda, which should you use to clean the wound? In what order?
	\choice{Water first, then soda}
	\choice[!]{Soda first, then water}
	\choice{Water only}
	\choice{All of the above}
\end{question}

\begin{question}
	How to care for an amputated organ?
	\choice{Immediately put in an ice bath.}
	\choice[!]{Wrap in clean cloth. Then submerge in ice.}
	\choice{Wrap in clean cloth. Put in a dry plastic bag. Lay on top of ice.}
	\choice{Put in a freezer}
\end{question}

\begin{question}
	How should a stabbing wound with an object stuck inside the body be treated?
	\choice{Remove the object. Then control the bleeding.}
	\choice{Remove the object. Insert a straw into the wound to drain the blood pooling inside the body.}
	\choice[!]{Wrap the object at the base with clean cloth to immobilize the object.}
	\choice{Hold the object tight with the patient's hand. Tie both the hand and the object to the body to immobilize.}
\end{question}

\begin{question}
	For an open wound with visible internal organs, what is the correct first-aid method?	
	\choice[!]{Cover loosely with clean cloth. The cloth should be slightly damped.}
	\choice{Push the organs back inside the body}
	\choice{Cover tightly with clean cloth}
	\choice{Open the wound and stuff the area with clean cloth}
\end{question}

\begin{question}
	If you cannot control the patient's bleeding after a pressure dressing has been used, what could you do?
	\choice[!]{Apply another layer of pressure dressing with clean cloth without removing the old layer.}
	\choice{Remove the old pressure dressing and apply a new one}
	\choice{There is nothing you can do. Transport to a hospital immediately.}
	\choice{Open the wound and stuff it with gauze}
\end{question}

\begin{question}
	Which is correct about a tourniquet?
	\choice{Always use a tourniquet. A tourniquet can stop a bleeding faster.}
	\choice[!]{Use for a large wound with a lot of bleeding or for a case in which other bleeding control methods do not work.}
	\choice{Use on any body parts}
	\choice{Use small objects to make a tourniquet. The tourniquet should be loosened briefly every 15 minutes.}
\end{question}

\begin{question}
	How to care for an amputated organ?
	\choice{Immediately put in an ice bath if a plastic bag cannot be found.}
	\choice{Put in a freezer. Wait for an ambulance.}
	\choice{Cover with clean cloth. Put in an ice bath.}
	\choice{Put in a clean container. Then put in ice.}
\end{question}

\begin{question}
	Which is \underline{incorrect} for seizure first-aid?
	\choice{After seizure, put the patient in the recovery position to prevent choking.}
	\choice[!]{Put a hard object in the patient's mouth to prevent the patient from biting down on his own tongue.}
	\choice{Do not restrain the patient}
	\choice{Keep record of the duration of the seizure.}
\end{question}

\begin{question}
	For an infant with high fever and seizure, what is the correct first-aid?
	\choice{Feed the infant with paracetamol or tylenol during the seizure.}
	\choice{Restrain the infant in thick blanket}
	\choice[!]{Wipe the patient down with damp cloth to reduce fever}
	\choice{All of the above are incorrect.}
\end{question}

\begin{question}
	Which is a factor for a quality CPR?
	\choice{Compress to 2-2.4 inches and fully release the pressure after each compression.}
	\choice{Compress 100-120 times a minute. Interruptions should be no longer than 10 seconds.}
	\choice{Hands are positioned on the nipple line or two fingers above the breastbone.}
	\choice[!]{All of the above}
\end{question}

\begin{question}
	Which patient needs a CPR?
	\choice[!]{20-yr old male. Faint. Unconscious. No breathing.}
	\choice{30-yr old female. In an accident. Unconscious. Bleeding with rapid pulses.}
	\choice{50-yr old male. Rapid breathing. Anxious. Unintelligible speech.}
	\choice{10-yr old child. Snake bite. Open his eyes when called. Shallow breathing.}
\end{question}

\begin{question}
	A 3-yr old boy drowned in a pool. The boy is removed from the pool. Unconscious and not breathing. You should...
	\choice{wipe down the patient. Cover with a thick blanket and transport the patient to a hospital.}
	\choice[!]{Call for an ambulance immediately. Perform CPR and use an AED when available.}
	\choice{Mouth to mouth breathing. Straddle the child across your should and jump up and down to remove the water from the boy's body.}
	\choice{Put the child on your shoulder. Jump up and down to remove the water from the boy's lungs.}
\end{question}

\begin{question}
	Which patient needs an AED?	
	\choice{Conscious with shallow breathing}
	\choice[!]{Drowned victim. Unconscious and not breathing.}
	\choice{Seizure patient. Unconscious with shallow breathing.}
	\choice{Alcohol-poisoned patient. Loud breathing.}
\end{question}

\begin{question}
	What is the correct order for a CPR?
	\choice[!]{30 compressions -> Open airway -> 2 rescue breaths}
	\choice{Open airway -> 30 compressions -> 2 rescue breaths}
	\choice{2 rescue breaths -> 30 compressions -> Open airway}
	\choice{Open airway -> 2 rescue breaths -> 30 compressions}
\end{question}

\begin{question}
	An unconscious patient. Not breathing. A lot of blood is found in the mouth. You should...
	\choice{Perform a CPR with 30 compressions and 2 rescue breaths using the mouth-to-mouth method.}
	\choice[!]{Perform a CPR using chest compression only. No rescue breaths.}
	\choice{Put in the recovery position to drain the blood from the mouth. Start a CPR with rescue breaths before chest compression.}
	\choice{Put in the recovery position. No CPR is necessary.}
\end{question}

\begin{question}
	When to use an AED?
	\choice{When the patient has a seizure and starts to breath again.}
	\choice{Each time a CPR has been performed for 2 minutes or 5 rounds.}
	\choice[!]{As soon as an AED is available.}
	\choice{When an ambulance has arrived.}
\end{question}

\begin{question}
	Which patient has the symptoms of a stroke?
	\choice{50-yr old Amy. Difficulty breathing}
	\choice{40-yr old Brian. Chest pain radiating to left arm.}
	\choice[!]{60-yr old Conan. Crippling headache. No strength in right arm. Drooped lip. Unintelligible speech.}
	\choice{30-yr old David. Stomach pain with diarrhea. Vomit.}
\end{question}

\begin{question}
	Which is the acronym for a stroke first-aid?
	\choice{S.A.M.P.L.E}
	\choice{D.R.S.A.B.C}
	\choice{A.B.C.D}
	\choice[!]{B.E.S.A.F.E}
\end{question}

\begin{question}
	What is the symptom that allows you to use the universal coverage for emergency patients? (UCEP is the rights to have an emergency medical treatment without having to pay.)
	\choice{Forceful, quick, and deep breathing. Loud breathing. Able to speak in short sentences only. Or unable to make a sound.}
	\choice{Conscious with altered state of mind. Unable to tell the time, place, or recognize familiar faces.}
	\choice{Intense chest pain. No strength in limbs. Seizure}
	\choice[!]{All of the above}
\end{question}

\begin{question}
	Fora patient that cannot breath and cannot make a sound. The patient's hands are on his throat. You should...
	\choice{Put the patient on your shoulder. Jump up and down.}
	\choice[!]{Approach the patient from behind. Clasp your hands between the the belly button and the breastbone. Give several forceful thrusts.}
	\choice{Transport to a hospital immediately. No action needed from you.}
	\choice{Wait until the patient becomes unconscious. Then perform a CPR.}
\end{question}

\begin{question}
	A 1-yr old girl swalled a coin. The coin blocks the airway. Wheezing when breath. Anxious. You should...
	\choice[!]{Give 5 back blows alternating with 5 chest compressions}
	\choice{Fish the coin out with your finger}
	\choice{Give forceful abdominal thrusts}
	\choice{Lie the patient down. Perform a CPR.}
\end{question}

\begin{question}
	Which is \underline{NOT} correct in treating a faint?
	\choice{Lie the patient down. Elevate the legs.}
	\choice[!]{Feed the patient water or medicine while the patient is unconscious.}
	\choice{Loosen clothing}
	\choice{Cordone off the area from bystanders to increase air circulation around the patient.}
\end{question}

\begin{question}
	How to treat a heat syncope (fainting from heat)?
	\choice{Lie down with elevated legs}
	\choice{Cover with a blanket to raise the body temperature}
	\choice[!]{Douse with water to reduce the body temperature}
	\choice{No action needed}
\end{question}

\begin{question}
	A patient with a severe allergic reaction to food. Swollen lips. Difficulty breathing. You should...
	\choice{Induce a vomit}
	\choice{Wait until the symptom is worse before transport to a hospital}
	\choice{Have the patient take medication immediately}
	\choice[!]{Go to the hospital immediately}
\end{question}

\begin{question}
	A patient got stung by a jellyfish. You should...
	\choice[!]{Wash with vinegar. Use a ruler to scrape out the mucus from the jelly fish.}
	\choice{Wash away the mucus with a lot of water}
	\choice{Rub with sand to remove the mucus}
	\choice{Use seaweed in the area to cover. Don't need to wash off the mucus.}
\end{question}

\begin{question}
	A snake bite wound with 2 fang marks. The snake species is unknown. You should...
	\choice{Suck on the wound with your mouth to remove the venom from the body.}
	\choice{Apply a tourniquet above the bitten area}
	\choice{Elevate the legs above the heart}
	\choice[!]{Wash the bitten area with water. Wrap the area with pressure dressing. Try not to move the bitten limb.}
\end{question}

\begin{question}
	Which patient should be given a serum?
	\choice{Every patient with a snake bite}
	\choice[!]{A patient that has a severe reaction to the bite}
	\choice{A patient that was bitten by an unknown type of snake.}
	\choice{A patient that was bitten but shows no reaction.}
\end{question}

\begin{question}
	Which is the correct fist-aid for a bee sting?
	\choice{Wrap with an elastic bandage to reduce the swelling}
	\choice[!]{Remove the stinger and apply an ice pack}
	\choice{Apply a hot pack to reduce the swelling.}
	\choice{None of the above}
\end{question}

\begin{question}
	How to treat a lionfish sting?
	\choice[!]{Soak in 45 Celsius water for 30-90 minutes}
	\choice{Pound the area with a shoe to dissipate the venom}
	\choice{Apply an ice pack}
	\choice{Tie above the area to prevent the venom from reaching the heart}
\end{question}

\begin{question}
	How to treat a sea urchin sting?
	\choice{Tie above the area to prevent the venom from reaching the heart}
	\choice{Pound the area with a shoe to break down the tentacle}
	\choice[!]{Squeeze lemon juice on the would or soak in warm water with vinegar}
	\choice{Soak in 45 Celsius water for 30 minutes}
\end{question}

\begin{question}
	Which is \underline{NOT} a factor for high blood pressure?
	\choice{Smoking}
	\choice[!]{Regular exercise}
	\choice{Overweight}
	\choice{Regularly consumption of fast food}
\end{question}

\begin{question}
	For an allergic reaction that is not severe and no medication is needed, you should...
	\choice{Rest}
	\choice{Exercise to promote long-term health}
	\choice{Prevent yourself from being exposed to known allergens}
	\choice[!]{All of the above}
\end{question}

\begin{question}
	What is the difference between a hyperventilation and an asthma?
	\choice{An asthma patient wheezes when breaths. No wheezing in a hyperventilation.}
	\choice{An asthma patient should not lie down. A hyperventilating patient can lie down.}
	\choice{An asthma patient stiffens their hands. A hyperventilating patient does not.}
	\choice[!]{All of the above}
\end{question}

\begin{question}
	For a patient with a headache. No strength in left arm. Difficulty speaking. What could be the cause and your action?
	\choice[!]{A stroke. Immediately go to a hospital within 4.5 hours}
	\choice{Myocardial infarction. Go to a hospital within 2 hours.}
	\choice{A vertigo. Lie down. No mediation needed.}
	\choice{A stroke. Rest and observe the symptom. If not better, see a doctor.}
\end{question}

\begin{question}
	Which choice are the possible symptoms for myocardial infarction?
	\choice{Soreness on left side of the body after tennis}
	\choice{Tightening pain in stomach. Episodic in nature for the past 2 weeks}
	\choice[!]{Chest pain that radiates up to the jaw. Heart fluttering. Easily tired}
	\choice{Pain under the breastbone when cough. Fever.}
\end{question}

\begin{question}
	A 30-yr old male. Fell from a tree. Conscious with back pain. Unable to move limbs. No wound. You should...
	\choice{Have the patient lie down. Elevated legs. Make a back splint.}
	\choice[!]{Lie down without a pillow. Don't support legs. Wait for an ambulance.}
	\choice{Put the patient in a recovery position. Open the airway.}
	\choice{Put the patient in a recovery position. Make a back splint.}
\end{question}

\begin{question}
	Which symptom is not severe enough to warrant a doctor's visit?
	\choice[!]{Diarrhea. Defecate 3 times in 1 day.}
	\choice{Vomit 10 times. Can't eat. Tired.}
	\choice{Cough. Fever. Thick saliva for the last 5 days.}
	\choice{Chest pain. Difficulty breathing. No strength}
\end{question}

\begin{question}
	Which patient could have brain damage?
	\choice{6-yr old child. Fell off a bed. Crying}
	\choice{10-yr old. Fell off a bike. Broken front teeth. Swollen forehead. Not crying because the child is afraid of being scolded.}
	\choice[!]{20-yr old male. Fell off a motorcycle. Not wearing a helmet. Briefly became unconscious. lethargic after woken up.}
	\choice{60-yr old male. Fell off a flight of stairs. Can't walk. Numb legs.}
\end{question}

\begin{question}
	Which country has the most recorded motor vehicle accidents?
	\choice[!]{Thailand}
	\choice{Vietnam}
	\choice{India}
	\choice{Laos}
\end{question}
\end{multiplechoice}

%\begin{endmatter}
%	\vfill
%	\vspace*{2in}
%	\begin{center}
%		{\huge This page is intentionally left blank.} \\
%	\vfill
%	\end{center}
%\end{endmatter}

\end{document}
