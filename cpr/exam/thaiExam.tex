\documentclass[a4paper,16pt]{examdesign}
% Using examdesign package from CTAN

\BoldfaceCorrectMultipleChoiceAnswer
\ShortKey
\NumberOfVersions{3}
\class{980-022 Basic Lifesaving}
\examname{Written test}
\setrandomseed{1024}
\def\namedata{}



\usepackage{fontspec}
\XeTeXlinebreaklocale "th_TH"
\defaultfontfeatures{Scale=2}
\setmainfont{TH Sarabun New}

\usepackage[margin=1in]{geometry}
\usepackage{enumitem}

\setlist[enumerate,3]{label=\Alph*}


\begin{document}


\begin{frontmatter}
	\vspace*{2in}
	\begin{center}
		{\huge 980-022 Basic Lifesaving} \\
		\vspace{1in}
		{\huge Written Test, ภาษาไทย} \\
		\vspace{1in}
		{\huge Version \framebox{\Alph{version}}} \\
		\vspace{1in}
		\begin{itemize}
		\item ข้อสอบนี้มีทั้งหมดรวม 11 หน้า กรุณาตรวจสอบว่าคุณได้รับทุกหน้า 
		\item ห้ามเขียนลงบนข้อสอบ ฝนคำตอบบนกระดาษคำตอบเท่านั้น
		\item ระบาย Version ของข้อสอบบนกระดาษคำตอบด้วย
		\item ข้อสอบใช้เวลา 1 ชั่วโมง
		\item ข้อสอบเป็นแบบปรนัย จำนวน 50 คำถาม
		\item กรุณาเช็คการฝน ID และ version ของข้อสอบบนกระดาษคำตอบก่อนส่ง
		\end{itemize}
	\end{center}
	\vfill
\end{frontmatter}


\begin{examtop}
	{\parbox{3in}{\classdata \\
	\examtype, Version: \fbox{\textsf{\Alph{version}}}}}
\end{examtop}

\begin{multiplechoice}[keycolumns=5]
\begin{question}
	ข้อใด\underline{ไม่ใช่}วัตถุประสงค์ในการปฐมพยาบาล
	\choice[!]{เพื่อรักษาให้ผู้ป่วยหายจากอาการเจ็บป่วยที่เป็นอยู่}
	\choice{เพื่อลดความเจ็บปวดของการบาดเจ็บ}
	\choice{เพื่อป้องกันความพิการ หรือความเจ็บปวดอื่น ๆ ที่จะเกิดขึ้นตามมาภายหลัง}
	\choice{เพื่อช่วยชีวิตผู้ป่วย หรือผู้ที่ได้รับบาดเจ็บจากเหตุการณ์หรืออุบัติเหตุต่าง ๆ ในขณะนั้น }
\end{question}

\begin{question}
	ข้อใดคือหมายเลขฉุกเฉินที่ใช้โทรเวลาเกิดเหตุทางด้านสุขภาพ
	\choice{1699}
	\choice[!]{1669}
	\choice{191}
	\choice{1724}
\end{question}

\begin{question}
	เราควรแจ้งรายละเอียดอะไรบ้างเวลาโทรขอความช่วยเหลือ (เป็นอย่างน้อย)
	\choice{สถานที่เกิดเหตุ เพศ อายุ สัญชาติ จำนวนผู้บาดเจ็บ}
	\choice{เกิดเหตุอะไร สถานที่เกิดเหตุ จำนวนผู้บาดเจ็บ หมายเลขติดต่อกลับ}
	\choice[!]{เกิดเหตุอะไร อาการของผู้บาดเจ็บ สถานที่เกิดเหตุ หมายเลขติดต่อกลับ}
	\choice{สถานที่เกิดเหตุ อาการของผู้บาดเจ็บ เวลาที่เกิดเหตุ หมายเลขติดต่อกลับ}
\end{question}

\begin{question}
	เมื่อพบผู้ป่วยเกิดอุบัติเหตุ ก่อนเคลื่อนย้ายควรทำอย่างไร เป็นลำดับแรก
	\choice[!]{ผู้บาดเจ็บทุกรายให้สงสัยไว้ก่อนว่ามีบาดเจ็บของกระดูกคอ และทำให้คออยู่ในแนวตรง}
	\choice{ถ้าผู้ป่วยอยู่กลางแจ้ง รีบเคลื่อนย้ายผู้ป่วยไปในที่ร่มทันที ไม่ต้องปฐมพยาบาล}
	\choice{ถ้าผู้ป่วยไม่รู้สึกตัวจัดท่านอนตะแคงกึ่งคว่ำ (ท่าพักฟื้น)}
	\choice{รีบยกเคลื่อนยายไป รพ. โดยไม่ต้องปฐมพยาบาลใดใด}
\end{question}

\begin{question}
	ข้อใดเป็นคำแนะนำที่เหมาะสมสำหรับผู้บาดเจ็บกระดูกหัก
	\choice{กรณีแขนหักให้ขยับมือเอาไว้ตลอดเพื่อให้มีการไหลเวียนของเลือด}
	\choice{กรณีกระดูกหักแบบเปิด ให้รีบดามกระดูกก่อนแล้วจึงปิดแผลห้ามเลือด}
	\choice[!]{กรณีกระดูกหัก ห้ามดื่มหรือกินอะไรเด็ดขาดก่อนไปถึงรพ.}
	\choice{การบาดเจ็บที่แค่บวมไม่มีการผิดรูปของกระดูก แปลว่าไม่มีการหักของกระดูกอย่างแน่นอน}
\end{question}

\begin{question}
	ผู้ป่วยมีแผลฟกช้ำ บวม ควรปฐมพยาบาลเบื้องต้นอย่างไร
	\choice[!]{ใช้ความเย็นประคบใน 24 ชม.แรกหลังเกิดอุบัติเหตุและความร้อนประคบหลังจาก 24 ชม.แรก}
	\choice{ใช้ความร้อนประคบใน 24 ชม.แรกหลังเกิดอุบัติเหตุและความเย็นประคบหลังจาก 24 ชม.แรก}
	\choice{ใช้ยานวดแบบร้อนทาบริเวณที่ฟกช้ำและบวม}
	\choice{บีบนวดคลึงเบา ๆ ที่รอยช้ำเพื่อให้หายเร็ว}
\end{question}

\begin{question}
	หลักการดามแขน หรือขา ที่สงสัยว่าจะมีกระดูกหักคือ
	\choice{จัดดึงกระดูกให้ข้ารูปแล้วพันด้วยผ้ายืดให้แน่นเพื่อลดบวม}
	\choice{ไม่ควรดามแขน หรือขา ที่สงสัยว่ากระดูกหักแบบเปิด เนื่องจากกลัวสิ่งสกปรกจะเข้าแผล}
	\choice[!]{ดามด้วยวัสดุที่ยาวเหนือและใต้ข้อต่อตำแหน่งที่กระดูกหัก เพื่อให้กระดูกอยู่นิ่ง}
	\choice{จัดให้กระดูกอยู่ในแนวตรงที่สุดแล้วรีบไปรพ.}
\end{question}

\begin{question}
	ไหล่หลุดปฐมพยาบาลอย่างไร
	\choice{พยายามดึงให้เข้าที่ให้ได้มากที่สุด}
	\choice{จัดให้แขนกางไว้ตลอดเวลา}
	\choice{ให้นอนพักเฉย ๆ สักพักจะเข้าที่เอง}
	\choice[!]{หลุดในท่าไหนให้อยู่ท่านั้น แล้วใช้ผ้าคล้องแขนให้อยู่นิ่ง ๆ}
\end{question}

\begin{question}
	การทำแผลโดนแทงที่ท้องและมีวัสดุปักคาข้อใดถูกต้อง
	\choice{ไม่ต้องทำแผล ให้ผู้บาดเจ็บจับวัสดุที่คาไว้ให้แน่นแล้วรีบนำส่ง รพ.}
	\choice{ดึงวัสดุออกแล้วรีบใช้ผ้าก้อสหนา ๆ กดแผลห้ามเลือดไว้}
	\choice[!]{ใช้ผ้าสะอาดห่อพันแผลให้วัสดุปักคาอยู่นิ่ง ๆ แล้วรีบนำส่งรพ.}
	\choice{ดึงออกแล้วหาหลอดมาใส่ในแผลเพื่อระบายเลือด}
\end{question}

\begin{question}
	ข้อใดถูกเกี่ยวกับการทำแผลไฟไหม้น้ำร้อนลวก
	\choice{กรณีที่แผลเป็นถุงน้ำขนาดใหญ่ให้ทำให้แตกแล้วเปิดแผลทิ้งไว้ให้แห้งหายไปเอง}
	\choice[!]{ใช้น้ำสะอาดราดบริเวณบาดแผลมาก ๆ นานอย่างน้อยสิบนาทีหรือจนกว่าอาการปวดแสบร้อนจะลดลง}
	\choice{ใช้ยาสีฟัน ขี้เถ้าหรือเกลือทาแผลทันที หลังโดนน้ำร้อนลวก}
	\choice{ใช้ยางว่านหางจรเข้ทาลงไปหนา ๆ}
\end{question}

\begin{question}
	เมื่อมีบาดแผลกระจกบาด ขนาดใหญ่เลือกไหลออกเยอะ เราทำการปฐมพยาบาลอย่างไร
	\choice{ใช้น้ำสะอาดราดล้างแผลเยอะ ๆ ก่อนปิดแผล}
	\choice{เปิดดูในแผลก่อนว่ามีสิ่งสกปรกตกค้างอยู่หรือไม่ก่อนปิดแผล}
	\choice{ใช้ผ้าสะอาดยัดลงไปในแผลให้แน่นเพื่อห้ามเลือด}
	\choice[!]{ใช้ผ้าสะอาดกดผิดแผลได้เลยโดยไม่ต้องล้างแผล}
\end{question}

\begin{question}
	ถ้าเจอผู้ประสบอุบัติเหตุตกคูน้ำเหยียบกระเบื้องมีแผลเลือดไหล และมีโคลนเต็มเท้า เรามีน้ำเปล่า น้ำอัดลม เราจะใช้อะไรล้างแผลได้บ้าง เรียงลำดับอย่างไร
	\choice{ใช้น้ำเปล่าล้างก่อน แล้วค่อยล้างด้วยน้ำอัดลม}
	\choice[!]{ใช้น้ำอัดลมล้างก่อน แล้วค่อยล้างด้วยน้ำเปล่า}
	\choice{ใช้น้ำเปล่าเพียงอย่างเดียว}
	\choice{ถูกทุกข้อ}
\end{question}

\begin{question}
	เราจะดูแลอวัยวะที่โดนตัดขาดอย่างไร
	\choice{รีบนำไปแช่ในน้ำแข็งทันที}
	\choice{ห่อผ้าสะอาดไว้ แล้วแช่ในน้ำแข็ง}	
	\choice[!]{ห่อด้วยผ้าสะอาดแล้วใส่ถุงพลาสติกที่แห้งมัดปาดถุงให้สนิทก่อนแช่ในน้ำแข็ง}	
	\choice{แช่ในช่องฟรีซ}
\end{question}

\begin{question}
	แผลที่มีวัสดุปักคาอยู่บริเวณหน้าอก ปฐมพยาบาลอย่างไร
	\choice{รีบดึงออกแล้วทำการห้ามเลือด}
	\choice{รีบดึงออกแล้วหาหลอดมาใส่ระบายเลือด}
	\choice[!]{ใช้ผ้าพันโคนแผลไว้ ให้วัสดุอยู่นิ่ง ๆ}
	\choice{ใช้มือจับวัสดุไว้ให้อยู่นิ่ง ๆ แล้วพันแผลพร้อมกับมือไว้รอบลำตัว}
\end{question}

\begin{question}
	แผลที่มีอวัยวะภายในไหลออกมาจากท้อง ปฐมพยาบาลอย่างไร
	\choice[!]{ใช้ผ้าสะอาดคลุมไว้ไม่ให้แน่น แล้วใช้น้ำสะอาดราดไว้พอชุ่ม}
	\choice{จับยัดกลับเข้าไปข้างในท้อง}
	\choice{ใช้ผ้าสะอาดกดปิดแผลให้แน่น}
	\choice{ไม่ต้องทำอะไร ปล่อยไว้เฉย ๆ}
\end{question}

\begin{question}
	หลังจากทำแผลห้ามเลือดปิดผ้าก็อซและพันแผลแล้วแต่ยังมีเลือดไหลซึมอีก เราจะห้ามเลือดอย่างไร
	\choice[!]{ใช้ผ้าก็อซปิดทับซ้ำหนาๆแล้วพันทับซ้ำได้เลยไม่ต้องแกะชิ้นเดิมออก}
	\choice{ให้แกะผ้าก็อซชิ้นที่ชุ่มเลือดออกก่อนแล้วค่อยปิดทำแผลใหม่}
	\choice{ไม่ต้องทำอะไรเพิ่ม ให้รีบนำส่ง รพ.ทันที}	
	\choice{เปิดแผลแล้วเอาผ้าก๊อซยัดลงในแผล}	
\end{question}

\begin{question}
	ข้อใดกล่างถูกต้องเกี่ยวกับการขันชะเนาะ (tourniquet)
	\choice{ทำในทุกแผลที่มีเลือดไหลเพราะเลือดจะได้หยุดไหลเร็วขึ้น}
	\choice[!]{ทำในกรณีแผลใหญ่ เลือดไหลเยอะ หรือห้ามเลือดด้วยวิธีอื่นไม่ได้ผลเท่านั้น}
	\choice{สามารถทำได้ทุกตำแหน่งของร่างกาย}
	\choice{ใช้วัสดุเส้นเล็ก ๆ เมื่อรัดแล้วควรคลายทุก 15 นาที}
\end{question}

\begin{question}
	เราสามารถเก็บอวัยวะที่โดนตัดขาดไว้อย่างไร
	\choice{แช่น้ำแข็งได้เลยถ้าไม่มีถุง}
	\choice{แช่ช่องฟรีซขณะที่รอรถพยาบาล}	
	\choice{ห่อกับผ้าก๊อซที่สะอาดแล้วแช่ในน้ำแข็ง}	
	\choice[!]{ใส่ในภาชนะใดก็ได้ที่สะอาด แล้วค่อยแช่ในน้ำแข็ง}
\end{question}

\begin{question}
	ข้อใด\underline{ผิด}ในการช่วยปฐมพยาบาลในผู้ป่วยชักเกร็ง (Seizure)
	\choice{หลังจากหยุดชักให้นอนตะแคงกึ่งคว่ำเพื่อป้องกันสำลัก}
	\choice[!]{ควรหาของแข็งหรือช้อนมางัดฟันป้องกันการกัดลิ้นขณะชัก}
	\choice{ไม่ควรจับมัดผู้ป่วยขณะชัก}
	\choice{จับเวลาของการชักแต่ละครั้ง}
\end{question}

\begin{question}
	เมื่อเด็กเล็กมีไข้สูงแล้วชัก เราควรให้การปฐมพยาบาลอย่างไร
	\choice{รีบป้อนยาลดไข้ทันทีในขณะที่ชัก}
	\choice{รีบมัดห่อตัวเด็กด้วยผ้าหน้า ๆ}	
	\choice[!]{รีบเช็ดตัวลดไข้ทันที}
	\choice{ผิดทุกข้อ}
\end{question}

\begin{question}
	ข้อใดคือคุณภาพของการ CPR
	\choice{กดหน้าอกลึก 2-2.4 นิ้ว และปล่อยมือให้หน้าอกคืนตัวสุด}
	\choice{กดหน้าอกเร็ว 100-120 ครั้งต่อนาที และหยุดกดไม่เกิน 10 วินาที}
	\choice{วางมือระหว่างราวนมหรือเหนือลิ้นปี่ 2 นิ้วมือ}
	\choice[!]{ถูกทุกข้อ}
\end{question}

\begin{question}
	ผู้ป่วยคนใดที่เราต้องทำ CPR
	\choice[!]{ชาย 20 ปี เป็นลม ไม่รู้สึกตัว ไม่หายใจ}
	\choice{หญิง 30 ปี อุบัติเหตุ สลบ มีแผลเลือดไหล ชีพจรเร็ว}
	\choice{ชาย 50 ปี หายใจเร็ว กระสับกระส่าย พูดคุยไม่รู้เรื่อง}
	\choice{เด็ก 10 ขวบโดนงูกัดเรียกลืมตา หายใจแผ่วเบา}
\end{question}

\begin{question}
	เด็กชายต่างชาติอายุ 3 ขวบ จมน้ำในสระ มีคนช่วยขึ้นมาจากน้ำ แต่ไม่รู้สึกตัว ไม่หายใจ ให้การช่วยเหลืออย่างไร
	\choice{รีบเช็ดตัว ห่อผ้าหนา ๆ แล้วรีบพาไปรพ.}
	\choice[!]{โทรศัพท์ขอรถพยาบาล ทำการ CPR และใช้เครื่อง AED เมื่อเครื่องมาถึง}
	\choice{ช่วยเป่าปาก และจับพาดบ่า กระทุ้งน้ำให้ออก}
	\choice{รีบจับพาดบ่า หรือกระทุ้งท้องเพื่อให้น้ำออกจากปอด}
\end{question}

\begin{question}
	เราจะใช้เครื่อง AED กับผู้ป่วยรายใด
	\choice{คนที่รู้สึกตัว หายใจแผ่วเบา}
	\choice[!]{คนที่จมน้ำ ไม่รู้สึกตัว ไม่หายใจ}
	\choice{คนที่ชักเกร็งกระตุก ไม่รู้สึกตัว หายใจเร็ว}
	\choice{คนเมาที่หมดสติ หายใจเสียงดัง}
\end{question}


\begin{question}
	จงเรียงลำดับขั้นตอนของการ CPR ที่ถูกต้อง
	\choice[!]{กดหน้าอก 30 ครั้ง -> เปิดทางเดินหายใจ -> ช่วยหายใจ 2 ครั้ง}
	\choice{เปิดทางเดินหายใจ -> กดหน้าอก 30 ครั้ง -> ช่วยหายใจ 2 ครั้ง}
	\choice{ช่วยหายใจ 2 ครั้ง -> กดหน้าอก 30 ครั้ง -> เปิดทางเดินหายใจ}
	\choice{เปิดทางเดินหายใจ -> ช่วยหายใจ 2 ครั้ง -> กดหน้าอก 30 ครั้ง}
\end{question}

\begin{question}
	ถ้าไปพบเจอผู้ป่วยหมดสติ ไม่รู้สึกตัว ไม่หายใจ มีเลือดเต็ปาก จะทำการช่วยเหลืออย่างไร
	\choice{CPR โดยการกดหน้าอก 30 ครั้ง แล้วช่วยหายใจ mouth to mouth 2 ครั้ง}
	\choice[!]{CPR โดยการกดหน้าอกอย่างเดียว ไม่ต้องเป่าปาก}
	\choice{จับนอนตะแคงล้างเลือดในปากก่อน แล้วจึง CPR โดยเริ่มจากเป่าปากแล้วไปกดหน้าอก}
	\choice{จัดท่านอนตะแคงเฉย ๆ ไม่ต้อง CPR}
\end{question}

\begin{question}
	เราจะใช้ AED เมื่อไหร่
	\choice{เมื่อผู้ป่วยหยุดชัก และเริ่มกลับมาหายใจ}
	\choice{เมื่อ CPR ครบ 2 นาที หรือ 5 รอบ}
	\choice[!]{ใช้ทันทีที่เครื่อง AED มาถึง}
	\choice{เมื่อเจ้าหน้าที่พยาบาลมาถึง}
\end{question}

\begin{question}
	ผู้ป่วยรายใดน่าจะมีอาการของภาวะหลอดเลือดสมอง ตีบ แตก ตัน
	\choice{นางแดง 50 ปี เหนื่อยหอบหายใจลำบาก}
	\choice{นายเขียว 40 ปี ใจสั่น บ่นเจ็บแน่นหน้าอก ร้าวไปแขนด้านซ้าย}
	\choice[!]{นายดำ 60 ปี ปวดศีรษะมาก แขนขวาอ่อนแรง ปากเบี้ยว พูดไม่ชัด}
	\choice{นายเหลือง 30 ปี ปวดท้อง ท้องเสีย อาเจียน}
\end{question}

\begin{question}
	ข้อใดคือตัวย่อของการประเมินภาวะหลอดเลือดสมองตีบ แตก ตัน (stroke)
	\choice{S.A.M.P.L.E}
	\choice{D.R.S.A.B.C}
	\choice{A.B.C.D}
	\choice[!]{B.E.F.A.S.T}
\end{question}

\begin{question}
	ข้อใดคืออาการฉุกเฉินที่สามารถใช้สิทธิ์ UCEP (สิทธิ์เข้ารับการรักษาได้ทุกรพ. ที่ใกล้ที่สุดเมื่อมีอาการเหล่านี้ โดยไม่ต้องเสียค่าใช้จ่าย)
	\choice{มีภาวะหายใจเร็ว แรง และลึก หายใจมีเสียงดังผิดปกติ พูดได้แต่สั้น ๆ หรือร้องไม่ออก ออกเสียงไม่ได้}
	\choice{มีการรับรู้ สติเปลี่ยนไป บอกเวลา สถานที่ คนที่คุ้นเคยผิดอย่างเฉียบพลัน}
	\choice{มีอาการเจ็บหน้าอกรุนแรง / แขนขาอ่อนแรงทันทีทันใด / ชักเกร็ง}
	\choice[!]{ถูกทุกข้อ}
\end{question}

\begin{question}
	เมื่อเจอคนมีอาการหายใจไม่ออก ส่งเสียงไม่ได้ มือกุมคอขณะทานอาหาร ควรช่วยเหลืออย่างไร
	\choice{จับพาดบ่ากระแทกทองแรง ๆ}
	\choice[!]{เข้าด้านหลัง วางมือระหว่างสะดือกับลิ้นปี่ แล้วกระทุ้งขึ้นแรง ๆ หลาย ๆ ครั้ง}
	\choice{รีบนำส่งรพ. โดยไม่ต้องช่วยเหลือใดใด}
	\choice{รอดูจนกว่าเขาจะแน่นิ่งแล้ว CPR}
\end{question}

\begin{question}
	เมื่อเจอเด็กประมาณ 1 ขวบ กลืนเหรียญ ติดคอ หายใจมีเสียงวี้ด ๆ กระสับกระส่าย จะให้การช่วยเหลืออย่างไร
	\choice[!]{ตบหลังระหว่างสะบักแรง ๆ 5 ครั้ง แล้วสลับกับการกดหน้าอก 5 ครั้ง}
	\choice{รีบใช้นิ้วล้วงในคอเอาเหรียญออก}
	\choice{กระทุ้งท้องแรง ๆ}
	\choice{จับนอนหงายแล้ว CPR}
\end{question}

\begin{question}
	ข้อได้\underline{ผิด}ในการปฐมพยาบาลคนเป็นลมหน้าซีดตัวเย็น
	\choice{ให้นอนราบยกขาสูงเล็กน้อย}
	\choice[!]{ให้ดื่มยาลมหรือน้ำหวานขณะที่ยังไม่รู้สึกตัว}
	\choice{คลายเสื้อผ้าให้หลวม}
	\choice{อย่าให้คนมุงเพื่อให้หายใจสะดวก}
\end{question}

\begin{question}
	เป็นลมแดดปฐมพยาบาลอย่างไร
	\choice{นอนราบยกขาสูง}
	\choice{ห่มผ้าให้ตัวอุ่นขึ้น}
	\choice[!]{ราดน้ำให้ชุ่มตัวเพื่อลดอุณหภูมิ}
	\choice{ปล่อยไว้เฉย ๆ จะหายเอง}
\end{question}

\begin{question}
	ถ้ามีนักท่องเที่ยวแพ้อาหารที่ทานรุนแรง ปากบวมหายใจไม่ออก จะให้การช่วยเหลืออย่างไร
	\choice{รีบล้วงคอให้อาเจียน}
	\choice{รอให้อาการลุกลามมาก ๆ ค่อยพาไปรพ.}
	\choice{ให้กินยาแก้แพ้ได้เลย}
	\choice[!]{รีบพาไป รพ. ทันที}
\end{question}

\begin{question}
	มีนักท่องเที่ยวโดนแมงกระพรุน ให้การปฐมพยาบาลอย่างไร
	\choice[!]{ราดด้วยน้ำส้มสายชู แล้วใช้ไม้บรรทัดครูดคลาบเมือกออก}
	\choice{ใช้น้ำเปล่าราดเยอะ ๆ จนคราบเมือกออกหมด}
	\choice{ใช้ทรายขยี้ให้คราบเมือกหลุดออก}
	\choice{ใช้ผักบุ้งทะเลพอก โดยไม่ล้างคราบเมือก}
\end{question}

\begin{question}
	ไม่มั่นใจว่างูชนิดใดกัดแต่มีรอยเขี้ยว 2 รอย ปฐมพยาบาลอย่าไร
	\choice{ใช้ปากดูดพิษออกทันที}
	\choice{รัดเหนือบาดแผลให้แน่น}
	\choice{ยกขาให้สูงกว่าระดับหัวใจ}
	\choice[!]{ล้างแผล ใช้ผ้ายืดพันขาแล้วดามไว้ให้อยู่นิ่ง ๆ}
\end{question}

\begin{question}
	ผู้บาดเจ็บที่โดนงูกัดรายใด ที่คิดว่าน่าจะได้รับเซรุ่ม
	\choice{ทุกรายที่สงสัยว่าโดนงูพิษ}
	\choice[!]{คนที่โดนงูพิษกัดแล้วมีอาการแพ้รุนแรง}
	\choice{รายที่ไม่แน่ใจว่างูอะไรกัด}
	\choice{รายที่โดนงูพิษกัดแต่ไม่มีอาการแพ้}
\end{question}

\begin{question}
	ข้อใดถูกต้องเกี่ยวกับการปฐมพยาบาลผึ้งต่อย
	\choice{ใช้ผ้ายืดพันทับลดอาการบวม}
	\choice[!]{ดึงเหล็กในออกแล้วประคบเย็น}
	\choice{ประคบร้อนเพื่อลดบวม}
	\choice{ไม่มีข้อถูก}
\end{question}

\begin{question}
	ควรปฐมพยาบาลอย่างไรเมื่อโดนเงี่ยงปลาสิงโต
	\choice[!]{แช่น้ำร้อน 45 เซลเซียส นาน 30-90 นาที}
	\choice{ทุบบริเวณแผลด้วยรองเท้าเพื่อให้พิษสลายตัว}
	\choice{ประคบเย็น}
	\choice{รัดเหนือแผลป้องกันพิษแล้วเข้าสู่หัวใจ}
\end{question}

\begin{question}
	เมื่อโดนแม่นทะเลตำเท้า จะปฐมพยาบาลอย่างไร
	\choice{รัดเหนือแผลป้องกันพิษเข้าสู่หัวใจ}
	\choice{ใช้รองเท้าทุบแรง ๆ ให้เข็มแตกให้ละเอียด}
	\choice[!]{บีบน้ำมะนาวใส่แผลหรือแช่น้ำร้อนผสมน้ำส้มสายชู}
	\choice{แช่น้ำร้อน 45 เซลเซียส นาน 30 นาที}
\end{question}

\begin{question}
	ข้อใด\underline{ไม่ใช่}ปัจจัยส่งเสริมให้เกิดโรคความดันโลหิตสูง
	\choice{สูบบุหรี่}
	\choice[!]{ออกกำลังกายบ่อย ๆ}
	\choice{ชอบทานอาหาร fast food}
	\choice{มีน้ำหนักตัวเยอะ}
\end{question}

\begin{question}
	เมื่อเป็นภูมิแพ้แต่อาการไม่รุนแรง ไม่ต้องการทานยา เราจะดูแลตัวเองอย่างไร
	\choice{พักผ่อนให้เพียงพอ}
	\choice{ออกกำลังกายให้ร่างกายแข็งแรง}
	\choice{เลี่ยงปัจจัยเสี่ยงที่ก่อให้เกิดอาการ}
	\choice[!]{ถูกทุกข้อ}
\end{question}

\begin{question}
	ข้อใดคือความแตกต่างของอาการหายใจเร็วเกิน (Hyperventilation) กับอาการของโรคหอบหืด (asthma)
	\choice{Asthma หายใจมีเสียงวี้ด Hyperventilation หายใจไม่มีเสียงวี้ด}
	\choice{Asthma มีอาการแล้วนอนราบไม่ได้ Hyperventilation มีอาการแล้วนอนราบได้}
	\choice{Asthma มีอาการแล้วมือไม่จีบเกร็ง Hyperventilation มีอาการแล้วมือจีบเกร็ง}
	\choice[!]{ถูกทุกข้อ}
\end{question}

\begin{question}
	เจอผู้ป่วยมีอาการ ปวดศีรษะ แขนข้างซ้ายไม่มีแรง พูดลำบาก สงสัยว่าเป็นโรคอะไรและปฐมพยาบาลอย่างไร
	\choice[!]{Stroke รีบนำส่ง รพ. ภายใน 4.5 ชม. ตั้งแต่เริ่มมีอาการ}
	\choice{โรคกล้ามเนื้อหัวใจตาย รีบส่ง รพ. ภายใน 2 ชม.}
	\choice{โรคบ้านหมุน ให้นอนพักไม่ต้องทานยาใดใด}
	\choice{Stroke ให้นอนพักดูอาการ ถ้าไม่ดีขึ้นค่อยไปหาหมอ}
\end{question}

\begin{question}
	อาการใดที่สงสัยว่าจะเป็นอาการของโรคกล้ามเนื้อหัวใจตาย
	\choice{ปวดเมื่อยไหล่ด้านซ้ายเวลาขยับหลังเล่นเทนนิส}
	\choice{ปวดเกร็ง บีบ ๆ ท้อง เป็น ๆ หาย ๆ มาเป็นเวลา 2 อาทิตย์}
	\choice[!]{แน่นกลางหน้าอกร้าวขึ้นมากราม ใจสั่นและเหนื่อยง่ายขึ้นเมื่อเดินขึ้นบันได}
	\choice{เจ็บใต้ชายโครงเวลาไอแรง ๆ และมีไข้}
\end{question}

\begin{question}
	ชายชาวต่างชาติอายุ 30 ปี ตกต้นไม้ รู้ตัวดี บ่นปวดหลังขาขยับไม่ได้ ไม่มีบาดแผล ให้การปฐมพยาบาลอย่างไร
	\choice{ให้นอนราบยกขาสูง หาไม้มาดามหลัง}
	\choice[!]{นอนราบไม่หนุนศีรษะ ไม่หนุนขา รอเจ้าหน้าที่มารับ}
	\choice{นอนคะแคงในท่านอนพักฟื้น เปิดทางเดินหายใจ}
	\choice{นอนคะแคงดามหลังด้วยไม้ยาว ๆ ไว้ให้ตรง}
\end{question}

\begin{question}
	อาการป่วยแบบใดที่ไม่จำเป็นต้องไปพบแพทย์
	\choice[!]{ท้องเสีย 3 ครั้งใน 1 วัน}
	\choice{อาเจียน 10 ครั้ง ทานอาหารไม่ได้ เพลีย}
	\choice{ไข้ ไอ เสมหะเขียวข้น เป็นมา 5 วัน}
	\choice{แน่นหน้าอก หายใจเหนื่อย ไม่มีแรง}
\end{question}

\begin{question}
	ผู้ป่วยรายใดที่สงสัยว่าน่าจะมีการบาดเจ็บของสมอง
	\choice{เด็กน้อย อายุ 6 เดือนกลิ้งตกเบาะที่นอน ร้องให้งอแง}
	\choice{เด็กอายุ 10 ขวบ ขี่จักรยานตกหลุม ฟันหน้าหัก หัวโน ไม่ร้อง เพราะกลัวโดนดุ}
	\choice[!]{ชายไทย 20 ปี ขี่มอเตอร์ไซด์ ไม่ใส่หมวกกันน็อค ล้มเอง สลบ ตื่นมาจำเหตุการณ์ไม่ได้ ง่วงซึมตลอดเวลา}
	\choice{ชายต่างชาติ 60 ปี ลื่นตกบันได ลุกเดินไม่ไหว ขาชา}
\end{question}

\begin{question}
	ประเทศใดมีสถิติการเกิดอุบัติเหตุบนท้องถนนมากที่สุด
	\choice[!]{ประเทศไทย}
	\choice{ประเทศเวียดนาม}
	\choice{ประเทศอินเดีย}
	\choice{ประเทศลาว}
\end{question}
\end{multiplechoice}

%\begin{endmatter}
%	\vfill
%	\vspace*{2in}
%	\begin{center}
%		{\huge This page is intentionally left blank.} \\
%	\vfill
%	\end{center}
%\end{endmatter}

\end{document}
